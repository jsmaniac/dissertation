\begin{schemeregion}
\label{chap:exp}

To determine whether Typed Scheme is practical and whether converting
PLT Scheme programs is feasible, we conducted a series of experiments
in porting existing Scheme programs of varying complexity to Typed
Scheme.

\section{Educational Code}
For small programs, which we expected to be written in a disciplined
style that would be easy to type-check, we turned to educational
code.  Our preliminary investigations and type system design indicated
that programs in the style of {\it How to Design Programs}~\cite{fffk:htdp}
would type-check successfully with our system, with only type
annotations required.  

To see how more traditional educational Scheme code would fare, we
rewrote most programs from  two additional text books: {\it The Little
  Schemer\/}~\cite{little-schemer} and {\it The Seasoned
  Schemer\/}~\cite{seasoned-schemer}.  Converting
these 500 lines of code usually required nothing but the declaration
of types for function headers.  The only difficulty encountered was
 an inability to express in our type system some
invariants on S-expressions that the code relied on.

Second, we ported 1,000 lines of educational code, which consisted of
the solutions to a number of exercises for an undergraduate
programming languages course.  Again, handing S-expressions proved the
greatest challenge, since the code used tests of the form
\scheme|(pair? (car x))|, which does not provide useful information to
the type system (formally, the filter of this expression is
$\epsilon|\epsilon$).  Typing such tests required adding new local bindings.
This code also made use of a non-standard datatype definition
facility, which required adaptation to work with Typed Scheme.

\section{Libraries}
We ported 500 lines of code implementing a variety of data
structures from S\o gaard~\nocite{galore}'s
\scheme|galore.plt| library package. While these data structures were
originally designed for a typed functional language, the
implementations were not written with typing in mind. Two sorts of
changes were required for typing this library. First, in several
places the library failed to check for erroneous input, resulting in potentially surprising
behavior.  Correcting this required adding tests for the erroneous
cases.  Second, in about a dozen places throughout the code,
polymorphic functions needed to be explicitly instantiated in order
for typechecking to proceed.  These changes were, again, in addition
to the annotation of bound variables.

\section{Applications}
Finally, a research intern ported two
sizable applications under the direction of the first author.  The
first was a 2,700 line implementation of a game, written in 2007, and
the second was a 500 line checkbook managing script, maintained for
 12 years.

The game is a version of the multi-player card game
Squadron Scramble.\footnote{Squadron Scramble resembles Rummy; it is
available from US Game Systems.}  The original implementation consists of
10 PLT Scheme modules, totaling 2,700 lines of implementation code,
including 500 lines of unit tests.  

A representative function definition from the game is given in figure
\ref{f:squad}.  This function creates a \scheme|turn| object, and hands it
to the appropriate \scheme|player|.  It then checks whether the game is
over and if necessary, constructs the new state of the game and returns it.

The changes to this complex function are confined to
the function header.  We have converted the original \scheme|define|
to \scheme|define:| and provided type annotations for each of the
formal parameters as well as the return type.  This function 
returns multiple values, as is indicated by the return type.  Other
than the header, no changes are required.  The types of all the
locally bound variables are inferred from their body of the individual
definitions.  

Structure types are used extensively in this example, as they are in
the entire implementation.  In the definition of the variables
\scheme|the-end| and \scheme|the-return-card|, occurrence typing is
used to distinguish between the \scheme|res| and \scheme|end|
structures.   

Some portions of the implementation required more effort to port to
Typed Scheme. For example, portions of the data used for the game is stored
in external XML files with a fixed format, and the program relies upon
the details of that format.  However, since this invariant is neither
checked nor specified in the program, the type system cannot verify it.  Therefore,
we moved the code handling the XML file into a separate, untyped module of fewer than 50 lines
that the typed portion uses via \scheme|require/typed|.


\begin{figure}
\schemeinput{squad.scm}
\caption{A Excerpt from the Squadron Scramble Game}
\label{f:squad}
\end{figure}

\section{Scripts}
The second application ported required similarly few changes.  This
script maintained financial records recorded in an S-expression
stored in a file.  The major change made to the program was the
addition of checks to ensure that data read from the file was in the
correct format before using it to create the relevant internal
data structures.  This was similar to the issue encountered with the
Squadron Scramble game, but since the problem concerned a single
function, we added the necessary checks rather than creating a new
module.  The other semantic change to the program was to maintain a
typing invariant of a data structure by construction, rather
than after-the-fact mutation.  As in the case of the Galore library,
we consider this typechecker-mandated change an improvement to
the original program, even though it has already been used successfully
for many years.
\end{schemeregion}
