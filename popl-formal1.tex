\def\cf{\ma{\metafont{combfilter}}}
\def\appf{\ma{\metafont{applyfilter}}}
\def\absf{\ma{\metafont{abstractfilter}}}

\def\constantTyping{
\[
\begin{array}{@{}l@{\ =\ }l@{\qquad}l}
  \dt{\numberp} & {\predty{\empt}{\num}} & \\
  \dt{\procp} & {\predty{\empt}{\proctop}} & \\
  \dt{\boolp} & {\predty{\empt}{\bool}} & \\
\ifpath
  \dt{\consp} & {\predty{\empt}{\consty{\top}{\top}}} & \\
\fi
  \dt{\addone} & {\simpleproc{\num}{\num}} & \\
  \dt{\notsym} & {\simpleproc{\top}{\bool}} & \\
  \dt{\zerop} & {\simpleproc{\num}{\bool}} & \\
\end{array}
\]
}

\def\Syntax{
\[
  \begin{altgrammar}
    \d{}, \e{}, \dots &::=& \x{}  \alt \comb{\e{}}{\movervec{\e{}}} 
    \alt \cond{\e{}}{\e{}}{\e{}} \ponly{\alt \cons{\e{}}{\e{}}} \alt \vv{}  &\mbox{Expressions} \\
    \vv{} &::=& \cc{}  \alt \tt \alt \ff \alt \n{} \alt \abs{\x{}}{\t{}}{\e{}} \ponly{\alt \cons{\vv{}}{\vv{}}}  & \mbox{Values} \\
\ifpath
    \cc{} &::=& \addone \alt \zerop \alt \numberp & \\
    & & \alt \boolp \alt \procp \alt \notsym  \alt & \\
    &   & \consp \alt \car \alt \cdr  & \mbox{Primitive Operations} 
    \vspace{1mm}
    \\
\else 
    \cc{} &::=& \addone \alt \zerop \alt \numberp & \\
    &~~~\alt& \boolp \alt \procp \alt \notsym  & \mbox{Primitive Operations} 
    \vspace{1mm}
    \\
\fi
    \E{} &::=& [] \alt 
    \marg
        {\comb{\overvec{\vv{}}}{\E{}\ {\overvec{\e{}}}}} % general multi-arg contexts
        {\comb{\E{}}{\e{}} \alt \comb{\vv{}}{\E{}}} % one-arg contexts
        \ponly{\alt \cons{\E{}}{\e{}} \alt \cons{\vv{}}{\E{}}} \alt \cond{\E{}}{\e{}}{\e{}} 
        & \mbox{Evaluation Contexts} \vspace{1mm} \\
  \end{altgrammar}
  \]
\hrule
\[
  \begin{altgrammar}
    \sig{},\t{} &::= & \top \alt \marg{\proctop \alt}{} \num \alt \ttt
    \alt \fft  
    \ponly{\alt \consty{\t{}}{\t{}}} \alt
         {\proctype {\movervec[i]{\t{}}} {\t{}} {\movervec[i]{\phih{}}} {\sh{}}} 
         \alt (\usym\ \overvec{\t{}})     &\hspace*{-0.03in}\mbox{Types} \\



         \p{} &::=&  \t{\wpi{\x{}}} \alt \compt{\wpi{\x{}}} \lonly{\alt \imp{\p{}}{\p{}}} \alt \bot & \hspace*{-0.03in}\mbox{Filters} \\

         \ph{} & ::=& \t{\ponly{\pi}} \alt \compt{\ponly{\pi}} \lonly{\alt \imp{\ph{}}{\ph{}}} \alt \both & \hspace*{-0.03in}\mbox{Latent Filters} \\

         \phii{} & ::= & \overvec{\p{}}|\overvec{\p{}} & \hspace*{-0.03in}\mbox{Filter Sets} \\

         \phih{} & ::= & \overvec{\ph{}}|\overvec{\ph{}} & \hspace*{-0.03in}\mbox{Latent Filter Sets} \\

         \s{} & ::= &   \wpi{\x{}} \alt \noeffect &\hspace*{-0.03in}\mbox{Objects} \\
         
         \sh{} & ::= & \pathmaybe{\pii} \alt \noeffect  & \hspace*{-0.03in}\mbox{Latent Objects} \\
         \ifpath
         \pi & ::= & \overvec{\pe{}} & \hspace*{-0.03in}\mbox{Paths}\\
         
         \pe{} & :: = & \pecar \alt \pecdr & \hspace*{-0.03in}\mbox{Path Elements} \\
         \else
         \fi
  \end{altgrammar}
  \]
}

\def\absrule{
\inferrule[T-Abs]
{\hastyeffphi [\mdel \G{},{\movervec[i]{\hastype{\x{}}{\sig{}}}}] {\e{}} {\t{}} {\phii{}} {\s{}} \\\\
\movervec[i]{\phih{} = \abstractfilter {\xi} {\phii{}}} \\\\
\sh{} = \left\{
 \protect{\begin{array}{ll}
  \pathmaybe{\pii}     & \text{if } \s{} = \wpi{\xi} \\
  \noeffect  & \text{otherwise}
 \end{array}} \right.
}
{\hastytrue
  {\abs[i]{\x{}}{\sig{}}{\e{}}} 
  {\proctype {\movervec[i]{\sig{}}} {\t{}} {\movervec[i]{\phih{}}} {\sh{}}}
}
}
\def\apprule{
\inferrule[T-App]
{ \hastyeffphi {\e{op}} {\t{op}}   {\phii{op}} {\s{op}}\\\\
  \movervec[i]{\hastyeffphi {\e{a}} {\t{a}}  {\phii{a}} {\s{a}}}\\\\ 
  \movervec[i]{\subtype {\t{a}}{\t{f}}} \\
  {\subtype {\t{op}} {\proctype{\movervec[i]{\t{f}}}{\t{r}}{\movervec[i]{\phih{f}}}{\sh{}}}} \\\\
\phii{r} = \movervec[i]{\applyfilter{\phih{f}}{\t{a}}{\s{a}}} \\\\
\s{r} = \left\{
  \protect{\begin{array}{ll}
      \wpip{\wpi{x}}     & \text{if } \sh{} = \pathmaybe{\pipi} \text{ and } {\sai} = \wpi{\x{}} \\
      \noeffect  & \text{otherwise}
    \end{array}} \right.
}
{\hastyeffphi {\comb{\e{op}}{\movervec[i]{\e{a}}}} {\t{r}} {\phii{r}} {\s{r}}}
}

\def\coreTypeRules{
\[
\inferrule[T-Var]{}{\hastyvar {\x{}} {\G{}(\x{})} {\x{}}}
\qquad
\inferrule[T-Const]{}{\hastytrue{\cc{}}{\dt{\cc{}}}}
\]
\[
\inferrule[T-True]{}{\hastytrue{\tt}{\ttt}}
\qquad
\inferrule[T-False]{}{\hastyfalse{\ff}{\fft}}
\qquad
\inferrule[T-Num]{}{\hastytrue{\n{}}{\num}} 
\]
\[
\absrule
\qquad
\apprule
\]
\[
\inferrule[T-If]
{
  \hastyeff{\e1}{\t1}{\op{+}}{\op{-}}{\s1}
  \\\\
  \hastyeffphi[\logic{\Delta \cup \op{+}, }{} \G{} + \logic{\opp1}{\op{+}}]{\e2}{\t2}{\phii2} {\s2}
  \\\\
  \hastyeffphi[\logic{\Delta \cup \op{-}, }{} \G{} + \logic{\opp2}{\op{-}}]{\e3}{\t3}{\phii3} {\s3}
  \\\\
  \subtype{\t2}{\t{}} \\
  \subtype{\t3}{\t{}} \\\\
  \phii{} = \combfilter{\op{+}|\op{-}}{\phii2}{\phii3}
}
{\hastyeffphi {\cond{\e1}{\e2}{\e3}} {\t{}}  {\phii{}} {\noeffect}}
\]
}

\def\Subtype{
\[
\inferrule[S-Refl]{ }{\subtype{\t{}}{\t{}}}
\ifmarg
\qquad\qquad
\inferrule[S-ProcTop]
{ }{\subtype {\proctype {\movervec[i]{\sig{}}} {\t{}} {\movervec[i]{\phih{}}} {\sh{}}} {\proctop}}
\else
\fi
\qquad
\inferrule[S-Top]{ }{\subtype{\t{}}{\top}}
\qquad
\ifpath
\inferrule[S-Pair]
{
  \subtype{\t1}{\t2} \\
  \subtype{\sig1}{\sig2}
}
{
  \subtype{\consty{\t1}{\sig1}}{\consty{\t2}{\sig2}}
}
\qquad
\else
\fi
\inferrule[S-Fun]
{
  \movervec[i]{\subtype{\sig{a}}{\t{a}}} \\ 
  \subtype{\t{r}}{\sig{r}} \\\\ 
  \movervec[i]{\phih{}' \subseteq \phih{}} \\ \sh{} = \sh{}' \text{ or } \sh{}' = \noeffect
}
{
  \subtype
  {\proctype{\movervec[i]{\t{a}}}{\t{r}}{\movervec[i]{\phih{}}}{\sh{}}}
  {\proctype{\movervec[i]{\sig{a}}}{\sig{r}}{\movervec[i]{\phih{}'}}{\sh{}'}}
}
\]
\[
\inferrule[S-UnionSuper]
{\exists i.\ {\subtype{\t{}}{\sig{i}}}}
{\subtype{\t{}}{(\usym\ \overvec[i]{\sig{}})}}
\qquad
\inferrule[S-UnionSub]
{\overvec{\subtype{\t{i}}{\sig{}}}}
{\subtype{(\usym\ \overvec{\t{}})}{\sig{}}}
\]
}

\newcommand{\af}[3]{\abstractfilter{#1}{#2} & {#3}}
\newcommand{\apf}[4]{\applyfilter{#1}{#2}{#3} & {#4}}

\def\FilterMeta{
$$
  \begin{array}{l@{\ =\ }l@{\qquad}l}
    \af{\x{}}{\op{+}|\op{-}}{\overvec{\abstractone{\x{}}{\p{+}}}|\overvec{\abstractone{\x{}}{\p{-}}}} &  \vspace{2mm}\\

    \abstractone{\x{}}{\bot}& {\both} & \\
    \abstractone{\x{}}{{\t{}}_{\wpi{\x{}}}}& {{\t{}}_{\wpi{}}} & \\
    \abstractone{\x{}}{\comp{\t{}}_{\wpi{\x{}}}}& {\comp{\t{}}_{\wpi{}}} & \\
    \abstractone{\x{}}{{\t{}}_{\wpi{\y{}}}}& {\empt} \text{ where } \x{} \neq \y{} \\
    \abstractone{\x{}}{\comp{\t{}}_{\wpi{\y{}}}}& {\empt} \text{ where } \x{} \neq \y{} \\
  \end{array}
$$
$$
  \begin{array}{l@{\ =\ }l@{\qquad}l}
    \apf{\oph{+}|\oph{-}}{\sig{}}{\s{}}{\overvec{\applyone{\ph+}{\sig{}}{\s{}}}|\overvec{\applyone{\ph-}{\sig{}}{\s{}}}} &
%    {\overvec[i]{\p{+_+}},\overvec[j]{\p{-_-}}|
%     \overvec[i]{\p{-_+}},\overvec[j]{\p{+_-}}} & 
%   \text{ where } \overvec[i]{\p{+_+},{\p{+_-}}}  = \overvec[i]{\applyone{\ph+}{\t{}}{\s{}}} \\
%   & & & \text{ and } \overvec[j]{\p{-_+},{\p{-_-}}}  = \overvec[j]{\applyone{\ph-}{\t{}}{\s{}}} \\
\vspace{2mm}\\
    \applyone{\both}{\sig{}}{\s{}} & \bot & \\
    \applyone{\comp{\t{}}_{{\pempt}}} {\sig{}} {\s{}} & \bot \text{   where } \subtype{\sig{}}{\t{}} &\\
    \applyone{{\t{}}_{{\pempt}}} {\sig{}} {\s{}} & \bot \text{   where } \nooverlap{\sig{}}{\t{}} &\\
    \applyone{\ph{}}{\sig{}}{\noeffect} & \empt & \\
    \applyone{\t{\ponly{\pi'}}} {\sig{}} {\wpi{\x{}}} & \t{{\wpip{\wpi{\x{}}}}} & \\
    \applyone{\comp{\t{}}_{\ponly{\pi'}}} {\sig{}} {\wpi{\x{}}} & \comp{\t{}}_{{\wpip{\wpi{\x{}}}}} & \\
   \end{array}
$$
}


\def\PathSyntax{
\[
  \begin{altgrammar}
    \e{} &::=& \dots \ponly{\alt \cons{\e{}}{\e{}}}  &\mbox{Expressions} \\
    \vv{} &::=& \dots \ponly{\alt \cons{\vv{}}{\vv{}}}  & \mbox{Values} \\
%\ifpath
    \cc{} &::=& \dots \alt \consp \alt \car \alt \cdr  & \mbox{Primitive Operations} \\
    \vspace{1mm}
    \E{} &::=& \dots  \ponly{\alt \cons{\E{}}{\e{}} \alt \cons{\vv{}}{\E{}}} 
        & \mbox{Evaluation Contexts} \vspace{1mm} \\
  \end{altgrammar}
  \]
\hrule
\[
  \begin{altgrammar}
    \sig{},\t{} &::= &   \dots \ponly{\alt \consty{\t{}}{\t{}}}      &\mbox{Types} \\

         \p{} &::=&   \t{\wpi{\x{}}} \alt \compt{\wpi{\x{}}} \lonly{\alt \imp{\p{}}{\p{}}} \alt \bot & \mbox{Filters} \\

         \ph{} & ::=&  \t{\ponly{\pi}} \alt \compt{\ponly{\pi}} \lonly{\alt \imp{\ph{}}{\ph{}}} \alt \both & \mbox{Latent Filters} \\

         \s{} & ::= &   \wpi{\x{}} \alt \noeffect &\mbox{Objects} \\
         
         \sh{} & ::= & \pathmaybe{\pii} \alt \noeffect  & \mbox{Latent Objects} \\
         \ifpath
         \pi & ::= & \overvec{\pe{}} & \mbox{Paths}\\
         
         \pe{} & :: = & \pecar \alt \pecdr & \mbox{Path Elements} \\
         \else
         \fi
  \end{altgrammar}
  \]
}


\def\pairRules{
\begin{displaymath}
  \inferrule[T-Cons]
  {
    \hastyeffphi{\e1}{\t1}{\phii1}{\s1} \\\\ \hastyeffphi{\e2}{\t2}{\phii2}{\s2} 
  }
  {
    \hastytrue{\cons{\e1}{\e2}} {\consty{\t1}{\t2}}
  }
\end{displaymath}

\begin{displaymath}
\inferrule[T-Car]
  {
    \hastyeffphi{\e{}} {\consty{\t1}{\t2}} {\phii{}} {\s{}} \\\\
    \phii{r} = \applyfilter{\comp{\fft}_{\pecar}|{\fft}_{\pecar}}{\consty{\t1}{\t2}} {\s{}} \\\\
    \s{r} = \left\{
      \protect{\begin{array}{ll}
          \pecar(\pi(x))     & \text{if } {\s{}} = \wpi{\x{}} \\
          \noeffect  & \text{otherwise}
        \end{array}} \right.
  }
  {
    \hastyeffphi{\comb{\car}{\e{}}}{\t1}{\phii{r}}{\s{r}}
  }
\end{displaymath}
\begin{displaymath}
\inferrule[T-Cdr]
  {
    \hastyeffphi{\e{}} {\consty{\t1}{\t2}} {\phii{}} {\s{}} \\\\
    \phii{r} = \applyfilter{\comp{\fft}_{\pecdr}|{\fft}_{\pecdr}}{\consty{\t1}{\t2}} {\s{}} \\\\
    \s{r} = \left\{
      \protect{\begin{array}{ll}
          \pecdr(\pi(x))     & \text{if } {\s{}} = \wpi{\x{}} \\
          \noeffect  & \text{otherwise}
        \end{array}} \right.
  }
  {
    \hastyeffphi{\comb{\cdr}{\e{}}}{\t2}{\phii{r}}{\s{r}} 
  }
\end{displaymath}
}

\def\logif{
\inferrule[T-If]
{
  \hastyeff{\e1}{\t1}{\op{+}}{\op{-}}{\s1}
  \\\\
  \lonly{\derive{\Delta}{\op{+}}{\opp{+}}}
  \\
  \lonly{\derive{\Delta}{\op{-}}{\opp{-}}}
  \\\\
  \hastyeffphi[\logic{\Delta \cup \op{+}, }{} \G{} + \logic{\opp{+}}{\op{+}}]{\e2}{\t2}{\phii2} {\s2}
  \\\\
  \hastyeffphi[\logic{\Delta \cup \op{-}, }{} \G{} + \logic{\opp{-}}{\op{-}}]{\e3}{\t3}{\phii3} {\s3}
  \\\\
  \subtype{\t2}{\t{}} \\
  \subtype{\t3}{\t{}} \\\\
  \phii{} = \combfilter{\op{+}|\op{-}}{\phii2}{\phii3}
}
{\hastyeffphi {\cond{\e1}{\e2}{\e3}} {\t{}}  {\phii{}} {\noeffect}}
}


\def\subpairrule{
\inferrule[S-Pair]
{
  \subtype{\t1}{\t2} \\
  \subtype{\sig1}{\sig2}
}
{
  \subtype{\consty{\t1}{\sig1}}{\consty{\t2}{\sig2}}
}
}

\def\opsem{
\[
\inferrule*[lab={E-Delta}]
        {\del{\cc{}}{\vv{}} = \vv{}'}
        {\step{\comb{\cc{}}{\vv{}}}{\vv{}'}}
\qquad\qquad
\inferrule[E-Beta]{}
      {\step{\comb{\abs[i]{\x{}}{\t{}}{\e{b}}}{\movervec[i]{\e{a}}}}{\subsmulti{\e{b}}{\x{}}{\e{a}}{i}}}
\]

\[
\inferrule[E-IfFalse]{}
      {\step{\cond{\ff}{\e2}{\e3}}{\e3}}
\qquad\qquad
\inferrule[E-IfTrue]{\vv{} \neq \ff}
      {\step{\cond{\vv{}}{\e2}{\e3}}{\e2}}
\]

\[
\inferrule*
        {\step{L}{R}}
        {\reduce{\E{}[L]}{\E{}[R]}}
\]

\[
\begin{array}{@{}l@{\ =\ }l@{\qquad}l}
  \del{\addone}{\n{}} & \n{} + 1 & \vspace{2mm}\\

  \del{\notsym}{\ff} & \tt & \\
  \del{\notsym}{\vv{}} & \ff & \text{ otherwise}\vspace{2mm}\\

  \del{\zerop}{0} & \tt & \\
  \del{\zerop}{\n{}} & \ff & \text{ otherwise} \vspace{2mm}\\

  \del{\numberp}{\n{}} & \tt & \\
  \del{\numberp}{\vv{}} & \ff & \text{ otherwise}\vspace{2mm}\\

  \del{\boolp}{\tt} & \tt & \\
  \del{\boolp}{\ff} & \tt & \\
  \del{\boolp}{\vv{}} & \ff & \text{ otherwise}\vspace{2mm}\\

  \del{\procp}{\abs{\x{}}{\t{}}{\e{}}} & \tt & \\
  \del{\procp}{\vv{}} & \ff & \text{ otherwise}\vspace{2mm}\\
\ifpath
  \del{\consp}{\cons{\vv1}{\vv2}} & \tt & \\
  \del{\consp}{\vv{}} & \ff & \text{ otherwise}\vspace{2mm}\\

  \del{\car}{\cons{\vv1}{\vv2}} & \vv1 & \\
  \del{\cdr}{\cons{\vv1}{\vv2}} & \vv2 & \\
\fi
\end{array}
\]
}

\def\deltaext{
\[
\begin{array}{@{}l@{\ =\ }l@{\qquad}l}
  \del{\consp}{\cons{\vv1}{\vv2}} & \tt & \\
  \del{\consp}{\vv{}} & \ff & \text{ otherwise}\vspace{2mm}\\

  \del{\car}{\cons{\vv1}{\vv2}} & \vv1 & \\
  \del{\cdr}{\cons{\vv1}{\vv2}} & \vv2 & \\
\end{array}
\]
}
\newcommand{\update}[3][\Gamma(\x{})]{\metafont{update}({#1},{#3}_{\ponly{#2}})}
\newcommand{\updatesimp}[2]{\metafont{update}({#1},{#2})}

\def\EnvPlusUpdate{
$$
\begin{array}{@{\Gamma\ +\ }l@{\ =\ }l@{\qquad}l}
  \t{\wpi{\x{}}},\op{} & (\Gamma,\x{}:\update{\pi}{\t{}}) + \op{}& \\
  \comp{\t{}}_{\wpi{\x{}}},\op{} & (\Gamma,\x{}:\update{\pi}{\comp{\t{}}}) + \op{}& \\
  \bot,\op{} & \Gamma' \hspace{.5in} \text{ where } {\forall \x{} \in \mathrm{dom}(\Gamma)}.\Gamma'(\x{}) = \bot\ &\\ %
  \empt & \Gamma & \\
%  \p{} & \Gamma,\x{}:\updatesimp{\Gamma(\x{})}{\p{}} & \\
%  \comp{\t{}}_{\wpi{\x{}}} & \Gamma,\x{}:\update{\pi}{\comp{\t{}}} & \\
%  \bot & \Gamma' & \text{ where } {\forall \x{}}.\Gamma'(\x{}) = \bot\ \\ %
\end{array}
$$
$$
\begin{array}{@{}l@{\ =\ }l@{\qquad}l}
\ifpath
  \update[\consty{\t{}}{\sig{}}]{\pi::\pecar}{\ups{}} & \consty{\update[\t{}]{\pi}{\ups{}}}{\sig{}}  & \\
  \update[\consty{\t{}}{\sig{}}]{\pi::\pecar}{\comp{\ups{}}} & \consty{\update[\t{}]{\pi}{\comp{\ups{}}}}{\sig{}}  & \\

  \update[\consty{\t{}}{\sig{}}]{\pi::\pecdr}{\ups{}} & \consty{\t{}}{\update[\sig{}]{\pi}{\ups{}}}  & \\
  \update[\consty{\t{}}{\sig{}}]{\pi::\pecdr}{\comp{\ups{}}} & \consty{\t{}}{\update[\sig{}]{\pi}{\comp{\ups{}}}}  & \\
\fi
  \update[\t{}]{\empt}{\ups{}} & \restrict{\t{}}{\ups{}}& \\

  \update[\t{}]{\empt}{\comp{\ups{}}} & \remove{\t{}}{\ups{}}& \\
%  \updatesimp{\t{}}{\both} & {\bot} & \\
\end{array}
%
$$
}

\def\RestrictRemoveOverlap{
$$
\begin{array}{@{}l@{\ =\ }l@{\qquad}l}
  \restrict{\t{}}{\sig{}} & \bot \hfill \text{ if } {\nooverlap{\t{}}{\sig{}}} \\
  \restrict{(\usym\ \overvec{\t{}})}{\sig{}} & (\usym\ \overvec{\restrict{\t{}}{\sig{}}})& \\
  \restrict{\t{}}{\sig{}} & \t{} \hfill \text{ if } {\subtype{\t{}}{\sig{}}} \\
  \restrict{\t{}}{\sig{}} & \sig{} \hfill \text{ otherwise} \vspace{5mm}\\
  \remove{\t{}}{\sig{}} & \bot \hfill \text{ if } {\subtype{\t{}}{\sig{}}} \\
  \remove{(\usym\ \overvec{\t{}})}{\sig{}} & (\usym\ \overvec{\remove{\t{}}{\sig{}}})& \\
  \remove{\t{}}{\sig{}} & \t{}  \hfill \text{  otherwise} \vspace{5mm}\\
  \nooverlap{\num}{\bool} & \mathrm{true} & \\
\ifpath
  \nooverlap{\num}{\consty{\t{}}{\sig{}}} & \mathrm{true} & \\
  \nooverlap{\bool}{\consty{\t{}}{\sig{}}} & \mathrm{true} & \\
\fi
  \nooverlap{\num}{\proctype{\movervec[i]{\t{}}}{\sig{}}{\movervec[i]{\phih{}}}{\sh{}}} & \mathrm{true} & \\
  \nooverlap{\bool}{\proctype{\movervec[i]{\t{}}}{\sig{}}{\movervec[i]{\phih{}}}{\sh{}}} & \mathrm{true} & \\
\ifpath
  \nooverlap
  {\consty{\t{}}{\sig{}}}
  {\proctype{\movervec[i]{\ups{a}}}{\ups{r}}{\movervec[i]{\phih{}}}{\sh{}}} & 
  \mathrm{true} & \\
\fi
  \nooverlap{(\usym\ \overvec{\t{}})}{\sig{}} & \bigwedge
  \overvec{\nooverlap{\t{}}{\sig{}}} & \\
  \nooverlap{\t{}}{\sig{}} & \mathrm{true} \qquad \hfill \text{ if } \nooverlap{\sig{}}{\t{}}\\
  \nooverlap{\t{}}{\sig{}} & \mathrm{false} \hfill \text{ otherwise}\\
\end{array}
$$
}



\begin{schemeregion}

\emph{Occurrence typing}, originally developed by
\citet{krcf:occurrence}, is the idea of assigning differing types to
different occurrences of variables based on the control flow of the
program.  In Typed Scheme, we use applications of type predicates such
\scheme|number?| to determine the types of variables, as shown in
chapter~\ref{chap:overview}. This chapter and the next formalize the
ideas of occurrence typing, as well as extensions to paths
(section~\ref{sec:paths}) and complex logical propositions
(section~\ref{sec:logic}), in the calculus \lts.  The statement of
type soundness and the necessary lemmas, along with proof sketches,
are given in section~\ref{sec:soundness}.

We begin our presentation of \lts with the base system.  The
fundamental judgement of the type system is:

\vspace{-4mm}
\large
\[
  \hastyeffphi{\e{}}{\t{}}{\phii{}}{\s{}}
\]
\normalsize
\noindent
It states that in type environment $\Gamma$, the expression \e{}
has type \t{}, following standard practice.  Additionally, the
judgement describes the filters \phii{} of \e{}, i.e., the
 propositions known to be true depending on whether \e{}
evaluates to true or false.  Finally, the object \s{} describes the
portion of the environment for which~\e{} in an accessor.

\section{Syntax and Operational Semantics}

\small
\begin{figure*}
\Syntax
\caption{Syntax of Terms and Types}
\label{fig:syntax}
\end{figure*}

\begin{figure*}
\opsem
\caption{Operational Semantics}
\label{fig:opsem}
\end{figure*}


\begin{figure*}
\coreTypeRules
\caption{Core Type Rules}
\label{fig:core}
\end{figure*}

\begin{figure*}
\Subtype
\caption{Subtyping}
\label{fig:sub}
\end{figure*}

\begin{figure*}
\FilterMeta
\caption{Filter Metafunctions}
\label{fig:meta}
\end{figure*}

\begin{figure*}
\EnvPlusUpdate
\caption{Environment Operations}
\label{fig:envop}
\end{figure*}
\normalsize


The syntax and semantics of \lts are standard, except for the
types; see figure~\ref{fig:syntax} and figure~\ref{fig:opsem}. 
Terms are either boolean constants, numbers, function
constants, variables, $\lambda$-abstractions or \scheme|if|
expressions.  All of these have the usual operational semantics, which
we therefore omit.  One distinctive feature of the
operational semantics is that values other than $\ff$ are treated
as true.

The distinctive features of \lts are found in the types.  A type is
either the top type $\top$, the number type \num, a type representing
either of the two constant booleans, a function type, or a (true) union 
type.

There are two distinctive aspects of these types.  First,
separating the booleans into particular types for true ($\ttt$) and
false ($\fft$) is necessary since many common Scheme idioms
 treat $\ff$ specially. The
more typical boolean type is $(\usym \ttt\ \fft)$, which is
abbreviated \bool. Second, function types come
with two annotations: a {\it latent filter set} \phih{} and a {\it latent
  object} \sh{}.

\subsection{Filters, Latent Filters, and Filter Sets}

Filters \p{} are  propositions
about types of variables.  The filters proved by a particular expression
form part of the type judgement for that expression.  A filter of the
form \t{\x{}} states that \x{} has type \t{}.  The filter
$\comp{\t{}}_{\x{}}$ states that \x{} does not have type \t{}.  The
filter $\bot$ represents impossibility.
%
Latent filters \ph{} take on analogous forms, but without attached
variables, and can be thought of as uninstantiated propositions.

A filter set \phii{} is a pair of sequences of filters, which is
written as $\op{+}|\op{-}$.  The first component, \op{+}, collects the
propositions that are true if the expression 
evaluates to a true value.
  The second, \op{-}, collects the propositions that are
true when the expression evaluates to false. Latent filter
sets \phih{} are the analogous, a pair of sequences of latent filters.
  Empty sequences are written $\empt$.  When $\bot$
(or the latent form $\both$)
is an element of a sequence of filters, we omit the other
elements, since they are irrelevant.

Consider the expression \scheme|(number? x)|, a prototypical example
of the use of predicates. It has the filter set
$\num_{\x{}}|\comp{\num}_{\x{}}$, which states that if it evaluates
to \scheme|#t|, then \x{} is a number, otherwise \x{} is not a
number.  The function \numberp has the
 latent filter $\num|\comp{\num}$ which
states that when applied to a value, if the result is true the
value is a number, and if the result is false, the value is
not a number.

\subsection{Objects and Latent Objects}

The second piece of additional information computed by the type system
is the \emph{object}.  It indicates for which part of the dynamic environment the
expression is an accessor.  In our base calculus,
 the only such expressions we consider are variable
references, which are given the variable itself as the object.  
Thus, the expression \x{} has object \x{}.  
The object $\noeffect$ indicates that no information is available
about what is accessed.\footnote{Objects are made richer in
  chapter~\ref{chap:occur-extend}.}  

Abstracting from an object results in a \emph{latent object}, which
represents the piece of the environment accessed when a function is
applied.  The latent object $\bullet$ indicates that whatever the
object of the argument to the function is, it is also the object of the
result. 
 The function $\abs{\x{}}{\t{}}{\x{}}$
has the latent object $\bullet$.
 The latent object $\noeffect$ indicates the absence of information
concerning the result.
The constant function $\abs{\x{}}{\t{}}{1}$
has latent object $\noeffect$, as does the function \numberp, whose
full type is
 $$\predty{\bullet}{\num}$$




\section{Typing Rules}

The typing rules for the base system of \lts are given in
figure~\ref{fig:core} (the $\cf$ metafunction is defined in
section~\ref{sec:cf}).
  The simplest rule is {\sc T-Num}, which gives
all numbers the type \num.  Since numbers are treated as true by the
evaluation rules for \scheme|if|, numbers have the filter set
$\empt|\bot$, indicating that no new information is acquired when the
number evaluates to a true value, and that if it evaluates to a false
value, a contradiction is obtained.  The rule for function constants,
{\sc T-Const}, is similar, but their types are given by the $\delta_{\t{}}$
meta-function, whose definition is given in figure~\ref{fig:dt}.  The
boolean constants are given singleton types in {\sc T-True} and {\sc
T-False}, along with filter sets that reflect that \scheme|#t| is
always true, and \scheme|#f| is always false.

The rule for typing variables involves additional aspects of the system.
The type of a variable is  looked up in the type
environment.  The object for a variable is itself.  Finally, the
filter for a variable indicates that if \x{} evaluates to a true value,
then \x{} cannot have type \fft.  Similarly, if \x{} evaluates
to \scheme|#f|, its type should also be \fft.

Typechecking an abstraction is straightforward.  The body is
typechecked in an appropriately-extended environment, producing a
type, filter set, and object.  The type is the result type of the
function, the filter set is abstracted into a latent filter set with
the \absf metafunction, defined in
figure~\ref{fig:meta}, and the object is abstracted if it is the
formal parameter of the abstraction.  The filter set and object of the
overall expression 
 is as for any other true value.

In the application rule, {\sc T-App}, all of the elements of the type
system are combined.  In addition to the standard checking of the
types of the formal and actual argument, information is propagated
from the operator's latent filter set and object and combined with the
object of the 
operand.  The \appf metafunction is (roughly) the
inverse operation of \absf, and explained further in
section~\ref{sec:cf}. 
Similarly,
the latent object of the operator is combined with the object of the
operand.  

Finally, in the {\sc T-If} rule, the filter set of the test
 is divided into its two components, and
the first is used for the \tbranch, the second for the 
\ebranch.  The operation $\Gamma + \op{}$, which combines a type
environment with filters to produce a new, refined type environment,
 is defined in
figure~\ref{fig:envop}.  This makes use of the metafunctions 
$\metafont{restrict}(\t{},\sig{})$ and $\metafont{remove}(\t{},\sig{})$,
 which intuitively compute  $\t{} \cap \sig{}$ and $\t{} - \sig{}$,
 respectively.  
%
The filter set for the entire expression is
computed from the filter sets of the three components by the
\cf metafunction, which is the subject of 
section~\ref{sec:cf}. 

\paragraph{Subtyping}
The subtyping relation on types is given in figure~\ref{fig:sub}.
Most of the rules are straightforward.  All types are subtypes of
$\top$, and function subtyping proceeds as usual, although
strengthening of the filter sets is permitted.  The rules for unions
are simple, and demonstrate that the union of no types, \scheme|(U)|,
 is below all other
types.  

\section{A Small Example}

Returning to example 1, we consider how the type system proves it
type safe. The example is \scheme|(if (number? x) (add1 x) 0)|, and
the  challenge is that the second occurrence of \x{} must be
given type \num. We assume that the type environment is $\Gamma =
\x{} : \top$. First, we consider the typing of the
application \scheme|(number? x)|.  The type of \x{} is $\top$, and its
object is just \x{}.  Second, the latent filter set of the
operand \scheme|number?| is $\num|\comp{\num}$, and the filter set of
the application is thus
$$\applyfilter{\num|\comp{\num}}{\top}{\x{}}
= \num_{\x{}}|\comp{\num}_{\x{}}$$

To typecheck the whole \scheme|if| expression, we take the relevant
filter set and add the first part to the type environment for the
\tbranch, giving $\x{} : \top + \num_{\x{}} = \x{} : \num$.  This
operation makes use of the $\metafont{update}$ and $\metafont{restrict}$
metafunctions, defined in figures~\ref{fig:envop} and \ref{fig:rro}. This
is sufficient to make \scheme|(add1 x)| typecheck correctly, and of
course the else expression  typechecks under any
environment.  Therefore, the whole expression has type \num, as
desired, with filter set $\empt|\empt$ and object $\noeffect$.  



\section{Manipulating Filters}
\label{sec:cf}

The metafunctions \appf, \absf, and \cf play a central role in the
functioning of the type system, and their definitions determine its
expressiveness.  The definitions of these metafunctions (and the
others defined in this chapter) are given in a ordered fashion,
i.e., each clause in their definitions (for example, in
figure~\ref{fig:meta})
is considered only if the preceding clauses do not apply.


\begin{itemize}
\item 
\absf, the simplest, maps a variable \scheme|x| and a filter set to
 a latent filter set.  In the case of a filter referring to
 \scheme|x|, it removes the variable, producing a latent filter. When it
encounters a filter that refers to a variable distinct from \scheme|x|,
that filter is discarded.  

\item
\appf is an inverse to \absf.  Given a latent filter set (from a
function operator) and a type and an object, it combines
them to produce a filter set.  In most cases, this simply replaces the
variable removed by \absf with the new variable.
  In cases where the combination of
the type and latent filter are impossible, $\bot$ is produced.  For
example, the expression \scheme|(number? 3)| has filter set
$\empt|\bot$ because $\applyfilter{\num|\comp{\num}}{\num}{\noeffect}$
 produces $\bot$ (since $\subtype{\num}{\num}$). Note that the
 $\nooverlap{\t{}}{\sig{}}$ metafunction determines if there are any values
 that have both type \t{} \emph{and} \sig{}.


\item
\cf allows for reasoning about the behavior of complex
\scheme|if| expressions.  The simplest definition for it is
$$%
\begin{array}{l@{\ =\ }l}
\combfilter{\phii1}{\phii2}{\phii3} & \empt|\empt  
\end{array}
$$


With this definition, nothing is known about the result of evaluating
a conditional expression.  However, more sophisticated definitions can
significantly increase the expressiveness of the system.  For example,
\scheme|(if #f $e_2$ $e_3$)| is equivalent to \scheme|$e_3$|.  
The following rules capture this intuition:
$$%
\begin{array}{l@{\ =\ }l}
\combfilter{\empt|\bot}{\phii2}{\phii3} & \phii2 \\
\combfilter{\bot|\empt}{\phii2}{\phii3} & \phii3  
\end{array}
$$

Since the compilation of many conditional and boolean operations in
 Scheme generate instances of \scheme|if|, \cf benefits from additional extensions.  For
example, \scheme|(and $e_1$ $e_2$)| expands to \scheme|(if $e_1$
$e_2$ #f)|.  As seen previously, if an \scheme|and| expression
produces a true value, then the filter sets of both subexpressions
should be combined, but if \scheme|#f| is produced, then nothing is
learned.  This is captured as follows:
$$%
\begin{array}{l@{\ =\ }l}
\combfilter{\op{1_+}|\op{1_-}} {\op{2_+}|\op{2_-}} {\bot|\empt} & 
\op{1_+},\op{2_+}|\logic{\imp{\op{1_+}}{\op{2_-}},\imp{\op{2_+}}{\op{1_-}}}{\empt}\\
\end{array}
$$%

As seen in example~\ref{ex:or}, uses of \scheme|or| can be combined
into filters that describe union types.  Here is an appropriate rule:

\newcommand{\ts}{\uniontwo{\t{}}{\sig{}}}


$$%
\begin{array}{l@{\ =\ }l}
{\combfilter 
{\t{\wpi{\x{}}}|\comp{\t{}}_{\wpi{\x{}}}}
{\empt|\bot}
{{\sig{\wpi{\x{}}}|\comp{\sig{}}_{\wpi{\x{}}}}} 
}
& \ts_{\wpi{\x{}}}|{\comp{\ts}}_{\wpi{\x{}}}  \\
\end{array}
$$

Additional rules are possible to handle \scheme|or| in greater
generality, as well as other patterns of \scheme|if| usage.
\end{itemize}



\small
\begin{figure}
\constantTyping
\caption{Constant Typing}
\label{fig:dt}
\end{figure}
\normalsize

\begin{figure}
\RestrictRemoveOverlap
\caption{Type Operations}
\label{fig:rro}
\end{figure}

\end{schemeregion}
