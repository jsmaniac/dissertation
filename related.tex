\begin{schemeregion}
Typed Scheme is  related to numerous other systems and
research programs.  In particular, many researchers have explored
connections between typed and untyped languages, with much recent work
under the heading of ``gradual typing''.  Typed Scheme also uses
numerous type system features pioneered in other languages.  Finally,
the particular innovative features of the type system, such as
occurrence typing and variable-arity polymorphism, are related to
other work on similar questions in other languages.   This chapter
provides a survey of this extensive literature.  

\section{Gradual Typing}

Under the name ``gradual typing'', several other researchers have
experimented with the integration of typed and untyped
code~\cite{st:gradual06, Herman07, wadler-findler, 
  ii:gradual-featherweight, stop09}.
  This work
has been pursued in two directions.  First, theoretical investigations
have considered integration of typed and untyped code at a much finer
granularity than we present, providing soundness theorems that prove
that only the untyped portions of the program can go wrong.  This is
analogous to earlier work on Typed Scheme~\cite{thf:dls2006},
which provides such a soundness theorem, given in chapter~\ref{chap:integrate}, which I believe scales to
full Typed Scheme and PLT Scheme. 

Second, \citet{furr:ruby:sac, furr:ruby:stop} have implemented a
system for Ruby which is similar to Typed Scheme.  They have also
designed a type system which matches the idioms of the underlying
language, and insert dynamic checks at the borders between typed and
untyped code.  Their work does not yet have a published soundness
theorem, and requires the use of a new Ruby interpreter, whereas Typed
Scheme runs purely as a library for PLT Scheme.

\citet{bracha:pluggable} suggests pluggable typing systems, in which a
programmer can choose from a variety of type systems for each piece of
code.  Although Typed Scheme requires some annotation, it can be thought of
as a step toward such a pluggable system, in which programmers can choose
between the standard PLT Scheme type system and Typed Scheme on a
module-by-module basis.

\section{Type System Features}

Many of the type system features in Typed
Scheme have been extensively studied.  Polymorphism in type systems
dates to \citet{reynolds}.  Recursive types are considered by
\citet{ac:subtyping-recursive-types}, and union
types by \citet{pierce-union}, among many others. 
My use of \lexeff{}s and especially \lateff{}s is inspired by prior
work on effect systems~\cite{fx87}.  

\section{Refinement Types}

Refinement types are due to \citet{fp:refinement}.
Since then, refinement types have been used in a wide variety of
systems~\cite{liquid-types,wadler-findler,f:hybrid}.  Previous
refinement type systems come in two varieties.  Freeman and Pfenning's
original system used the underlying language of ML types to specify
subsets of the existing types, such as non-empty lists.  Most other
systems have paired predicates in some potentially-restricted language
with a base type, meaning the set of values of that base type accepted
by that predicate.  Typically, this requires some algorithm for
deciding implication between predicates for subtyping.  In some
languages, this can be an external and almost always incomplete theorem
prover, as in the Liquid Typing~\cite{liquid-types} and Hybrid
Typing~\cite{f:hybrid} approaches.  

The approach taken by Typed Scheme differs from both of these
approaches.  First, refinements are not specified using the language
of data constructors but as in-language predicates.  This
allows any computable set to be a refinement.  Second,
no attempt is made to decide implication between predicates.  Two
distinct functions might be extensionally equivalent, but the
associated refinement types have no subtyping relationship.  This
frees both the programmer and the implementor from the burden of
 a theorem prover.  

\section{Types and Logic}

Considering types as logical propositions has a long history,
going back to Curry and Howard~\cite{cf:cl, Howard80}.  In a
dependently typed language such a Coq~\cite{COQBook} or Agda~\cite{agda}, the
relationships we describe with filters and objects could be encoded in
types, since types can contain arbitrary Coq terms, including ones
that reference other variables or the expression itself.  

In Typed Scheme, the rule for typechecking \scheme|if| expressions
propagates information known when the test evaluates
to true to the then branch, and information known when the test
evaluates to false to the else branch.  This could be expressed with
an \verb|if| combinator of the following Coq type:

\begin{verbatim}
  forall Q1 Q2 Q P, (P = true -> Q1) -> (P = false -> Q2) 
   -> (Q1 -> Q) -> (Q2 -> Q) -> Q
\end{verbatim}
\noindent
In a language with dependent types, this combinator could be used with
decision procedures that supply witnesses, a style already used in
Coq.  Procedures such as \scheme|number?| would be reinterpreted as
such decision procedures.  

From this perspective, the Typed Scheme type system
carves out a subset of the logical reasoning process available in 
 programming languages with such rich type systems.  This subset is tailored to the idioms and
styles of existing untyped Scheme code bases, and has allowed Typed Scheme to
typecheck a wide variety of existing Scheme code.

\section{Occurrence Typing}

The term ``occurrence typing'' was coined by~\citet{krcf:occurrence}
in their work on automatic understanding of legacy COBOL programs.
Their system considers a limited
form of occurrence typing, restricted to equality tests between
variables and character constants, which are used as tag checks in
existing COBOL programs.  Their system treats such
equalities as propositions, and negates them in the else branch.  It
does not allow abstractions over predicates, more general forms of
predicates, or tests on non-variables. 

Some features similar to those of occurrence typing have appeared in the dependent
type literature. \citet{c:typed-lisp} describes Typed Lisp, which includes
{\texttt{typecase}} expression that refines the type of a variable in the
various cases; \citet{weirich98} re-invent this construct in the context of
a typed lambda calculus with intensional polymorphism.  The
{\texttt{typecase}} statement specifies the variable to be refined, and
that variable is typed differently on the right-hand sides of the
\texttt{typecase} expression.  While this system is superficially similar
to occurrence typing in Typed Scheme, the use of latent and visible predicates allows us
to handle cases other than simple uses of \texttt{typecase}.  This is important
in type-checking existing Scheme code, which is not written with
\texttt{typecase} constructs. 
Intensional
polymorphism without refinement of the types of variables appears in
calculi by \citet{harper95compiling}, among others. 
\begin{schemeregion}
\section{Variable-Arity Polymorphism}
\label{sec:esop-related}

Variable-arity functions are nearly ubiquitous in the world of programming
languages, but no typed language supports them in a systematic and
principled manner.  Here we survey existing systems as well as several
theoretical efforts.

ANSI C provides ``varargs,'' but the functions that
implement this functionality serve as a thin wrapper around direct
access to the stack frame.  Java~\cite{jls3} and \csharp{} are two
statically typed languages that have only uniform variable-arity
functions, since access occurs via a homogeneous array.

\citet{dh:var-ar} come close to our goal of providing a
practical type system for variable-arity functions.  As part of the
Infer system for type-checking Scheme~\cite{h:infer}, they use an
encoding of ``infinitary tuples'' as row types for an ML-like type
inference system that handles optional arguments and uniform and
non-uniform variable-arity functions.

In comparison to the Typed Scheme approach, Dzeng and Haynes' system has several
limitations.  Most importantly, since their system does not support
first-class polymorphic functions, they are unable to type many of the
definitions of variable-arity functions, such as \scheme|map| or
\scheme|fold|.  Additionally, their system requires full type
inference to avoid exposing users to the underlying details of row
types, and it is also designed around a Hindley-Milner style
algorithm. This renders it incompatible with the remainder of the
design of Typed Scheme, which is based on a system with subtyping.

\citet{cpp-varity} propose an extension for
variadic templates to \cpp{} for the upcoming \cpp{}0x standard.  This
proposal has been accepted by the \cpp{} standardization committee.
Variadic templates provide a basis for implementing non-uniform
variable-arity functions in templates.  Since the approach is grounded
in templates, it is difficult to translate their approach to other
languages without template systems.  
The template approach addresses a simpler problem because template expansion
 is a pre-processing step and types are only checked after template expansion. 
It also significantly complicates the language,
 since arbitrary computation can be performed during template expansion.
Further, the template approach prevents checking of variadic functions
at the definition site, meaning that errors in the definition are only
caught when the function is used.

\citet{zip-calc} attempts to bring non-uniform variable-arity
functions to Haskell via the Zip Calculus, a type system with
restricted dependent types and special kinds that serve as tuple
dimensions.  This work is theoretical and  comes without
practical evaluation.  The presented limitations of the Zip
Calculus imply that it cannot assign a variable-arity type to the definition of 
\Verb|zipWith| (Haskell's name for Scheme's \scheme|map|) without
further extension, whereas Typed Scheme can do so.

Similarly, \citet{faking-it} and \citet{moggi} present
restricted forms of dependent typing in which the number of arguments
is passed as a parameter to variadic functions.  Our system, while not
allowing the expression of every dependently-typable program, is
simpler than dependent typing, suffices for most examples we have
encountered, and does not require an extra function parameter.

\end{schemeregion}


\section{Types and Flow Analysis}

In addition to the work on soft typing discussed in
chapter~\ref{chap:prior}, other flow analysis research has attempted to
solve problems similar to those that occurrence typing addresses.  

In chapter
9 of his thesis~\cite{shivers-thesis}, Shivers describes a type
recovery analysis that includes refining the type of variables in type
tests.  However, this only covered the particular case of tests for
integers, and did not include a general mechanism for handling
arbitrary type tests, or abstraction over such tests.  

The conditional types of \citet{awl:popl94} are closely related to
occurrence typing. Conditional types rely on a 'case' expression with
patterns. These patterns can include nested patterns, allowing
inspection of portions of compound data structures, as in Typed
Scheme.  However, conditional types do not support abstraction over
predicates, e.g., in Typed Scheme the expression \scheme|(lambda ([x :
Any]) (or (number? x) (string?  x)))| has a non-trivial latent filter,
whereas patterns cannot be abstracted over. Furthermore, conditional
typing does not propagate information about possible values matched in
one pattern to subsequent case branches.  For example, in contrast to
occurrence typing, conditional typing cannot determine for `case x of
cons(true,b) : true , cons(a,b) : false' that `a' cannot be `true' in
the second branch.

\subsection{Deriving Propositions}

Some flow analysis systems derive formulae that describe the program that are
not simply richer types.  For example, Might's Logic Flow
Analysis~\cite{might:lfa} derives propositions about the equality of
numeric values during the run of a flow analysis. The overall design of
Might's system is different, since it involves an external
theorem prover in interaction with a flow analysis, and derives
propositions that are radically different from mine.  However, in a type
checking setting, we expect that Might's system would be able to
derive similar theorems to those of \lts.  
\end{schemeregion}
