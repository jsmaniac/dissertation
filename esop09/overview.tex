\begin{schemeregion}
\section{An Introduction to  Variable-Arity Functions}
\label{sec:overview-va}

A Scheme programmer defines functions with \scheme{lambda} or
 \scheme{define}. Both syntactic forms support fixed and variable-arity
 parameter specifications:
\begin{enumerate}
\item \scheme{(lambda (x y) (+ x (* y 3)))} creates a function of two
arguments and \scheme{(define (f x y) (+ x (* y 3)))}
creates the same function and names it \scheme{f};

\item the function \scheme{(lambda (x y . z) (+ x (apply max y z)))}
consumes at least two arguments and otherwise as many as needed; 

\item \scheme{(define (g x y . z) (+ x (apply max y z)))} names this
function \scheme{g};

\item \scheme{(lambda z (apply + z))} creates a function of arbitrary
arity; and

\item \scheme{(define (h . z) (apply + z))} is the analogue to this
\scheme{lambda} expression. 
\end{enumerate}
 The parameter \scheme{z} in the last four cases is called the \emph{rest
 parameter}. 

The application of a variable-arity function combines any arguments in
 excess of the number of required parameters into a list.  Thus, \scheme{(g
 1 2 3 4)} binds \scheme{x} to \scheme{1} and \scheme{y} to \scheme{2},
 while \scheme{z} becomes \scheme{(list 3 4)} for the evaluation of
 \scheme{g}'s function body. In contrast, \scheme{(h 1 2 3 4)} sets
 \scheme{z} to \scheme{(list 1 2 3 4)}.

The \scheme|apply| function, used in the examples above, takes a
function \scheme|f|, a sequence of fixed arguments $v_1 \ldots v_n$,
and a list of additional arguments $r$.  If the list $r$ is the value
\scheme|(list w$_1$ $\ldots$ w$_n$)|, then \scheme|(apply f v$_1$
$\ldots$ v$_n$ r)| is the same as \scheme|(f v$_1$ $\ldots$ v$_n$ w$_1$
$\ldots$ w$_n$)|.  The \scheme|apply| function plays a critical role in
conjunction with rest arguments.


This section sketches how the revised version of Typed Scheme accommodates
 variable-arity functions. Our revision focuses on the uses of
 such functions that accept arbitrarily many arguments. Scheme
 programmers sometimes use variable-arity functions to simulate
 optional or keyword arguments. In PLT Scheme, such programs typically employ
 \scheme{case-lambda}~\cite{dybvig-case-lambda} or equivalent \scheme{define}
 forms instead. 


\subsection{Uniform Variable-Arity Functions}
\label{ssec:uni-va}

\emph{Uniform} variable-arity functions are those that expect their rest parameter to be a
 homogeneous list.  Consider the following three examples of type signatures:
\begin{schemedisplay}
(: + (Integer$^*$ -> Integer))
(: string-append (String$^*$ -> String))
(: list (All (alpha) (alpha$^*$ -> (Listof alpha))))
\end{schemedisplay}
 The syntax \scheme{Type$^*$} for the type of rest parameters alludes to
 the Kleene star for regular expressions.  It signals that in addition to
 the other arguments, the function takes an arbitrary number of arguments
 of the given base type.  The form \scheme|Type$^*$| is dubbed a {\em
 starred pre-type}, because it is not a full-fledged type and may appear
 only as the last element of a function's domain.

Here is a definition of variable-arity \scheme|+| in
Scheme:
%
\schemeinput{esop09/examples/plus.ss}

\noindent Typing this definition is straightforward. The
 type assigned to the rest parameter of starred pre-type \scheme|tau$^*$|
 in the body of the function is \scheme|(Listof tau)|, a
 pre-existing type in Typed Scheme.

\subsection{Beyond Uniform Variable-Arity Functions}
\label{ssec:beyond-uva}

Not all variable-arity functions assume that their rest parameter is a
homogeneous list of values.  We can allow heterogeneous rest
parameters by finding other constraints.  For example, the length of
the list assigned to the rest parameter may be connected to the types of
other parameters or the returned value.

For example, Scheme's \scheme|map| function
is not restricted to mapping unary functions over single lists,
 unlike its counter-parts in ML or Haskell. When
\scheme{map} receives a function \scheme{f} and $n$ lists, it expects
\scheme{f} to accept $n$ arguments.  Also, the type of the $k$th function
parameter must match the element type of the $k$th list.

 Scheme's \scheme|apply| function provides its own
challenges.  It is straightforward to type the use of \scheme|apply|
with a uniform variable-arity function, as in the hypothetical
definition of \scheme|+| from section~\ref{ssec:uni-va}.  If the type
of \scheme|f| involves the starred pre-type \scheme|tau$^*$|, then the
list $r$ must have type \scheme|(Listof tau)|.

However, take the following example taken from the PLT Scheme code base:
\begin{schemedisplay}
;; implements a wrapper that prints \scheme{f}'s arguments
(define (verbose f)
  (if quiet? f (lambda a (printf "xform-cpp: ~a\n" a) (apply f a))))
\end{schemedisplay}
The intent of the programmer is clear---the result of applying
\scheme|verbose| to a function \scheme|f| should have the same type as
\scheme|f| for {\em any} function type.  With uniform
variable-arity functions, we can type the internal \scheme|lambda|'s
argument \scheme|a| only as a homogeneous list of arbitrary length.
Thus, if \scheme|f| has some fixed arity $n$, then there is no way to
statically guarantee that the list of arguments $a$ has $n$ elements,
and thus applying the function \scheme|f| to the list \scheme|a| via
\scheme|apply| may go wrong.  Our type system should protect the
programmer from arity mismatches, whether through function application
or uses of \scheme|apply|, while allowing functions like \scheme|verbose|.

\subsection{Non-Uniform Variable-Arity Functions}

As of the latest release, Typed Scheme can represent the types of \emph{non-uniform} variable-arity
functions.  Below are the types for some example functions:
\begin{schemedisplay}
;; \scheme{map} is the standard Scheme map
(: map 
   (All (gamma alpha beta ...) 
     ((alpha beta dotsb -> gamma) (Listof alpha) (Listof beta) dotsb -> (Listof gamma))))

;; \scheme{map-with-funcs} takes any number of functions, 
;; and then an appropriate set of arguments, and then produces 
;; the results of applying all the functions to the arguments
(: map-with-funcs
   (All (beta alpha ...) ((alpha dotsa -> beta)$^*$ -> (alpha dotsa -> (Listof beta)))))
\end{schemedisplay}
Our first key innovation is the possibility to attach \scheme|...| to the last
type variable in the binding position of a \scheme|All| type constructor.
Such type variables are dubbed {\em dotted type variables}.  Dotted type
variables signal that this polymorphic type can be instantiated with an arbitrary
number of types.

Next, the body of \scheme|All| types with dotted type variables may contain
 expressions of the form \scheme|tau dotsa| for some type \scheme|tau| and
 a dotted type variable \scheme|alpha|.  These are {\em dotted pre-types};
 they classify non-uniform rest parameters just like starred pre-types classify
 uniform rest parameters.  A dotted pre-type has two parts: the base $\tau$
 and the bound $\alpha$.  Only dotted type variables can be used as the
 bound of a dotted pre-type. Since \scheme|All|-types are
 nestable and thus multiple dotted type variables may be in scope,
 dotted pre-types must specify the bound.

When a dotted polymorphic type is instantiated, any dotted pre-types
are expanded by copying the base an appropriate number of
times and by replacing free instances of the bound in each copy
with the corresponding type argument.  For example, instantiating
\scheme|map-with-funcs| as follows:
%% \scheme|Number Integer Boolean String|
\begin{schemedisplay}
(inst map-with-funcs Number Integer Boolean String)
\end{schemedisplay}
 results in a value with the type:
\begin{schemedisplay}
((Integer Boolean String -> Number)$^*$ ->
(Integer Boolean String -> (Listof Number)))
\end{schemedisplay}

Typed Scheme also provides local inference of the appropriate type
arguments for dotted polymorphic functions, so explicit type
instantiation is rarely needed~\cite{dots-tr}. Thus, the following uses
of \scheme|map| are successfully inferred at the appropriate types:
\begin{schemedisplay}
(map not (list #t #f #t))
;; \scheme{map} is instantiated (via local type inference) at:
;; \scheme|((Boolean -> Boolean) (Listof Boolean) -> (Listof Boolean))|
 
(map make-book (list "Flatland") (list "A. Square") (list 1884))
;; \scheme|((String String Integer -> book)|
;; $\ $\scheme|(Listof String) (Listof String) (Listof Integer) -> (Listof book))|
\end{schemedisplay}

Typed Scheme can also type-check the definitions of non-uniform
variable-arity functions: 
\schemeinput{esop09/examples/fold-left.ss}
The example introduces a definition of
 \scheme|fold-left|. Its type shows that it accepts at least three arguments: a
 function \scheme{f}; an initial element \scheme{c};
 and at least one list \scheme{as}. Optionally,
 \scheme{fold-left} consumes another sequence \scheme{bss} of lists. For
 this combination to work out, \scheme{f} must consume as many arguments as
 there are lists plus one; in addition, the types of these lists must match
 the types of \scheme{f}'s parameters because each list item 
 becomes an argument. 

Beyond this, the example illustrates that the rest parameter is treated as
 if it were a place-holder for a plain list parameter. In this particular
 case, \scheme{bss} is thrice subjected to \scheme{map}-style processing.
 In general, variable-arity functions should be free to process their rest
 parameters with existing list-processing functions. 

The challenge is to assign types to such expressions. On the one hand,
 list-processing functions expect lists, but the rest parameter has a
 dotted pre-type. On the other hand, the result of list-processing a rest
 parameter may flow again into a rest-argument position. While the first
 obstacle is simple to overcome with a conversion from dotted
 pre-types to list types, the second one is onerous. Since
 list-processing functions do not return dotted pre-types but list types,
 we cannot possibly expect that such list types come with enough
 information for an automatic conversion. 

Thus we use special type rules for the list
processing of rest parameters with \scheme|map|, \scheme|andmap|, and
\scheme|ormap|.  Consider \scheme{map}, which returns a list of the
 same length as the given one and whose component types are in a predictable
 order.  If \scheme|xs| is classified by the dotted pre-type
 \scheme|tau dotsa|, and \scheme|f| has type \scheme|(tau -> sigma)|, we classify
 \scheme|(map f xs)| with the dotted pre-type \scheme|sigma dotsa|.
 Thus, in the definition of \scheme{fold-left} \scheme|(map car bss)|
 is classified as the dotted pre-type \scheme|beta dotsb| because
 \scheme{car} is instantiated at \scheme|((Listof beta) -> beta)| and
 \scheme|bss| is classified as the dotted pre-type \scheme|(Listof beta)
 dotsb|.

One way to use such processed rest parameters is in conjunction with
 \scheme{apply}. Specifically, if \scheme{apply} is passed a variable-arity
 function \scheme{f}, then its final argument \scheme{l}, which must be a
 list, must match up with the rest parameter of \scheme{f}.  If the
 function is a uniform variable-arity procedure and the final argument is a
 list, typing the use of \scheme|apply| is straightforward. If it is a
 {\em non-uniform} variable-arity function, the number and types of parameters
 must match the elements and types of \scheme{l}.

Here is an illustrative example from the definition of \scheme|fold-left|:
\begin{schemedisplay}
(apply f c (car as) (map car bss))
\end{schemedisplay}
 By the type of
 \scheme{fold-left}, \scheme|f| has type \scheme|(gamma alpha beta dotsb ->
 gamma)|.  The types of \scheme|c| and \scheme|(car as)| match the types of
 the initial parameters to \scheme|f|.  Since the \scheme{map} application
 has dotted pre-type \scheme{(Listof beta) dotsb} and since the rest parameter
 position of \scheme|f| is bounded by \scheme|beta|, we are guaranteed that
 the length of the list produced by \scheme|(map car bss)| matches \scheme|f|'s
 expectations about its rest argument.  In short, we use
 the type system to show that we cannot have an arity mismatch,
 even in the case of \scheme|apply|.

\end{schemeregion}
