\begin{schemeregion}

\section{Types for Variable-Arity Functions}
\label{sec:intro}

For the past two years, Tobin-Hochstadt and Felleisen~\cite{thf:dls06,thf:popl08} have been
 developing Typed Scheme, an explicitly and statically typed sister
 language of PLT Scheme~\cite{mf:mzscheme}. In many cases, Typed Scheme
 accommodates existing Scheme programming idioms as much as possible.  One
 remaining obstacle concerns functions of variable arity. Such functions
 have a long history in programming languages, especially in LISP and
 Scheme systems where they are widely used for a variety of purposes,
 ranging from arithmetic operations to list processing. In response, we have
 augmented Typed Scheme so that its type system can cope
 with variable-arity functions of many kinds.

Some variadic functions in Scheme are quite simple. For example, the
 function \scheme{+} takes any number of numeric values and produces their
 sum.  This function, and others like it, could be typed in a system that
 maps a variable number of arguments into a homogeneous data
 structure.\footnote{Languages like C, C++, Java, and C\# support
 such variable-arity functions.} Other variable-arity functions,
 however, demand a more sophisticated approach than collecting
 the extra arguments in such a fashion.

Consider Scheme's \scheme{map} function, which takes a function as input as well
 as an arbitrary number of lists.  It then applies the function to the
 elements of the lists in a pointwise fashion.  The function must therefore
 take precisely as many arguments as the number of lists provided.  For
 example, if the \scheme|make-student| function consumes two arguments, a name as a string and
 a number for a grade, then the expression
\begin{schemedisplay}
(map make-student (list "Al" "Bob" "Carol") (list 87 98 64))
\end{schemedisplay} 
 produces a list.  We refer to variable-arity functions such as
 \scheme|+| and \scheme|map| as having \emph{uniform} and
 \emph{non-uniform} polymorphic types, respectively.

For Typed Scheme to be useful to working programmers, its type system must
 handle this form of polymorphism.  Further, although \scheme{map} and \scheme{+}
 are part of the standard library, language implementers cannot arrogate
 the ability to abstract over the arities of functions. Scheme programmers
 routinely define such functions, and if they wish to refactor their
 Scheme programs into Typed Scheme, our language must allow such
 function definitions. 

Of course, our concerns are relevant beyond the confines of Scheme.
Variable-arity functions are also useful in statically typed
languages, but are barely supported because of a lack of pragmatic
approaches.  Even the standard libraries of highly expressive typed
functional languages contain functions that would benefit from
non-uniform variable-arity, but instead are defined via copying of
code.  For example, the SML Basis Library~\cite{sml-basis} includes
the \Verb|ARRAY| and \Verb|ARRAY2| signatures, which include functions
that differ only in arity.  The GHC standard library~\cite{manual:ghc}
also features close to a dozen families of functions (such as
\Verb|zip| and \Verb|zipWith|) defined at a variety of arities.  We
conjecture that if their type systems provided variable-arity
polymorphism, Haskell and ML programmers would routinely define such
functions, too.

In this paper, we present the first pragmatic and comprehensive approach
 to variable-arity polymorphism: its design, implementation, and evaluation.
% new from MF
Our new version of Typed Scheme can assign types to hundreds of
programmer-introduced function definitions with variable arities,
something that was simply impossible before. Furthermore, we can now
type check library functions such as \scheme|map| and
\scheme|fold-left| without resorting to special tricks or duplication.

 In the next two sections, we describe
 Typed Scheme in general terms and then present the type system for
 variable-arity functions.  In
 section \ref{sec:formal}, we introduce a formal model of our
 variable-arity type system.  % In section \ref{sec:infer}, we describe how
% our innovations are integrated with the rest of Typed Scheme and sketch a
% local type inference algorithm.
 In section \ref{sec:eval} we present
 preliminary results of our evaluation effort with
 respect to the PLT Scheme code base and the limitations of our system.  In section
 \ref{sec:related} we discuss related work.

\end{schemeregion}
