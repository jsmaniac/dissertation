\begin{schemeregion}

\section{Evaluation}
\label{sec:eval}

Mining the extensive PLT Scheme code base
 provides significant evidence that variable-arity functions are
frequently defined and used; examining a fair number of examples shows
that our type system is able to cope with a good portion of
these definitions and uses.

\subsection{Measurements of Existing Code}

A simple pattern-based search of the code base for definitions of
 variable-arity functions and uses of certain built-in core functions
 produces the following:
\begin{itemize}
\item There are at least 1761 definitions of variable-arity functions.  

\item There are 488 uses of 
  \scheme|map|, \scheme|for-each|, \scheme|foldl|, \scheme|foldr|,
  \scheme|andmap|, and \scheme|ormap| with more than the minimum number
  of arguments.

\end{itemize}
 
These numbers demonstrate the need for a type system that deals
 with variable-arity functions. Programmers use those from the core
 library at multiple arities. Furthermore, programmers define
 such functions regularly.

It is this kind of inspection of our code base that inspires a careful
 investigation of the issue of variable-arity functions. We cannot reasonably
 ask our programmers to duplicate their code or to duplicate type cases
 just because our type system does not accommodate their utilization of the
 expressive power of plain Scheme. 

\subsection{Evaluation of Examples}

Simply counting definitions and uses of variable-arity functions is
 insufficient.  Each definition and use demands a separate inspection in order
 to validate that our type system can cope with it. This is particularly
 necessary for function definitions, because our pattern-based search does
 not indicate whether a definition introduces a uniform or non-uniform
 variable-arity function. 

\paragraph{Uses} The sample set for uses of variable-arity functions from
 the core library covers 30 cases, i.e., 10 randomly-chosen example
 function applications using each of \scheme|map|, \scheme|for-each|, and
 \scheme|andmap| with at least two list arguments.  For \scheme|map|, we are
 able to type 9 of 10, for \scheme|for-each| we are able to type 10 of 10,
 for \scheme|andmap| we are able to type 10 of 10.

In short, our technique is extremely successful for the list-processing
 functions, checking 29 of the 30 examples.  The one failure is due to the
 use of a list to represent a piece of information that comprises four
 pieces. In this case, our type system simply does not preserve the
 length information for the list from the input to \scheme|map|.

\paragraph{Definitions} The sample set for definitions of variable-arity
 functions covers some 120 cases (or some 7\%) from the code base. Our
 findings naturally sort these samples into three categories:
\begin{itemize}
\item A majority of the functions can be typed with uniform rest
  arguments or use variable arity to simulate optional arguments. For
  the latter, we recommend that programmers rewrite such functions
  using \scheme{case-lambda}.

\item Twelve of the 120 inspected definitions are non-uniform and require
  variable-arity polymorphism. Our type checker can assign types to
  all of them.  Returning to our example in section~\ref{ssec:beyond-uva},
  \scheme|verbose| can be given the type
\begin{schemedisplay}
  (All (beta alpha ...) ((alpha dotsa -> beta) -> (alpha dotsa -> beta))).
\end{schemedisplay}

\item The small remainder cannot be typed using our system.
\end{itemize}

These inspections demonstrate two important points. First,
 all of the various ways in which Typed Scheme handles varying numbers of
 arguments are important for type-checking existing Scheme
 code. Second, our design choices for variable-arity polymorphism mostly
 capture the programming style used in practice by working PLT Scheme
 programmers.
%
In conclusion, we conjecture that our type system can validate more
than 95\% of the uses of heterogeneous library functions such as \scheme|map|, that it can
check 70\% of the close to 1800 definitions, 10\% of which require the
heterogeneous version of variable-arity polymorphism.

%%% \subsection{Limitations and Future Work}

%%% Typed Scheme currently suffers from two major limitations for the
%%%  definition of non-uniform variable-arity functions. First, some functions
%%%  perform calculations that involve the length of the rest argument.  Our
%%%  type system cannot verify that such operations are type correct. 

%%% %% Also, it is possible to implement optional arguments with variable-arity
%%% %% functions, but our system cannot type such functions.

%%% Second, there are several type constructors where use of \scheme|...| would
%%%  make sense, but which we have not yet investigated, e.g., \scheme|(U alpha
%%%  dotsa)| or \scheme|(List alpha dotsa)|, where the \scheme|List| type
%%%  constructor represents fixed length, heterogeneous lists. Doing so would
%%%  help with Scheme functions that deal with fixed-length lists. 

%% Finally, our investigation into the code base reveals an open problem
%% concerning the \scheme{compose} function. Naturally, in Scheme
%% \scheme{compose} computes the composition of an arbitrary number of
%% functions and deals with multiple values. Our type system easily deals with
%% the full-fledged two-argument version: 
%% \begin{schemedisplay}
%% (: compose 
%%  (All (alpha ...) 
%%   (All (gamma beta ...)
%%    ((alpha dotsa -> (Values beta dotsb))
%%     (beta dotsb -> gamma)
%%     ->
%%     (alpha dotsa -> gamma)))))
%% \end{schemedisplay}

%% Even without considering multiple-value return, however, we know of no type
%% system that can type-check Scheme's variable-arity \scheme|compose|.
\end{schemeregion}
