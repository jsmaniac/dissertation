\begin{schemeregion}
\section{Variable-Arity Polymorphism}
\label{sec:esop-related}

Variable-arity functions are nearly ubiquitous in the world of programming
languages, but no typed language supports them in a systematic and
principled manner.  Here we survey existing systems as well as several
theoretical efforts.

ANSI C provides ``varargs,'' but the functions that
implement this functionality serve as a thin wrapper around direct
access to the stack frame.  Java~\cite{jls3} and \csharp{} are two
statically typed languages that have only uniform variable-arity
functions, since access occurs via a homogeneous array.

\citet{dh:var-ar} come close to our goal of providing a
practical type system for variable-arity functions.  As part of the
Infer system for type-checking Scheme~\cite{h:infer}, they use an
encoding of ``infinitary tuples'' as row types for an ML-like type
inference system that handles optional arguments and uniform and
non-uniform variable-arity functions.

In comparison to the Typed Scheme approach, Dzeng and Haynes' system has several
limitations.  Most importantly, since their system does not support
first-class polymorphic functions, they are unable to type many of the
definitions of variable-arity functions, such as \scheme|map| or
\scheme|fold|.  Additionally, their system requires full type
inference to avoid exposing users to the underlying details of row
types, and it is also designed around a Hindley-Milner style
algorithm. This renders it incompatible with the remainder of the
design of Typed Scheme, which is based on a system with subtyping.

\citet{cpp-varity} propose an extension for
variadic templates to \cpp{} for the upcoming \cpp{}0x standard.  This
proposal has been accepted by the \cpp{} standardization committee.
Variadic templates provide a basis for implementing non-uniform
variable-arity functions in templates.  Since the approach is grounded
in templates, it is difficult to translate their approach to other
languages without template systems.  
The template approach addresses a simpler problem because template expansion
 is a pre-processing step and types are only checked after template expansion. 
It also significantly complicates the language,
 since arbitrary computation can be performed during template expansion.
Further, the template approach prevents checking of variadic functions
at the definition site, meaning that errors in the definition are only
caught when the function is used.

\citet{zip-calc} attempts to bring non-uniform variable-arity
functions to Haskell via the Zip Calculus, a type system with
restricted dependent types and special kinds that serve as tuple
dimensions.  This work is theoretical and  comes without
practical evaluation.  The presented limitations of the Zip
Calculus imply that it cannot assign a variable-arity type to the definition of 
\Verb|zipWith| (Haskell's name for Scheme's \scheme|map|) without
further extension, whereas Typed Scheme can do so.

Similarly, \citet{faking-it} and \citet{moggi} present
restricted forms of dependent typing in which the number of arguments
is passed as a parameter to variadic functions.  Our system, while not
allowing the expression of every dependently-typable program, is
simpler than dependent typing, suffices for most examples we have
encountered, and does not require an extra function parameter.

\end{schemeregion}
