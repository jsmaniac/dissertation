\begin{schemeregion}

Most aspects of the implementation of Typed Scheme are standard fare
for typed programming languages.  However, there is one key novelty:
the implementation is done entirely in terms of PLT Scheme macros.
This implementation choice provides the following advantages: 

\begin{itemize}
\item Running the program runs the typechecker---there is no separate
  checker to run, as there is with many static checkers for Scheme.
\item By expanding into untyped PLT Scheme code, integration with
  untyped code is seamless and requires no translation.
\item The typechecker integrates relatively smoothly with the rest of the macro
  and module system.
\item The typechecker can take advantage of existing PLT Scheme
  infrastructure. 
\end{itemize}

Implementing a typechecker as a macro provides its own
challenges.  In particular, the system must deal with the existence of
other macros, with cross-module interaction, and it must be able
to communicate information about the source program to the
typechecker, even though the expander is oblivious to the type system.  

The macro and module system in PLT Scheme is uniquely well-suited for
the implementation of Typed Scheme.  It contains a multitude of small
and large features, whose development has been guided by the goal of
supporting research on and development of new languages.  Even though
they may be of marginal use individually, together they form a
comprehensive language implementation framework.

This chapter first introduces the relevant prerequisites of the PLT Scheme macro
system (section~\ref{sect:macros}),
its integration with the module system
(section~\ref{sect:modules}), as well as how different macros can communicate
with each other (section~\ref{sect:protocols}); 
Typed Scheme's implementation uses all of these tools.
  The implementation is first described relative to a
single module (section~\ref{sect:type-one}), and then for the multi-module
case
(section~\ref{sect:type-multi}). 

% \section{Syntax Framework}
% \label{sect:syntax}

% The evolution of PLT Scheme has been guided by two major goals:
% \begin{itemize}
% \item implementing DrScheme and its teaching languages
% \item supporting research on programming tools and languages
% \end{itemize}
% These goals have pushed PLT Scheme to develop a powerful framework for
% representing and manipulating programs.

% This section presents a summary of the standard and nonstandard
% features of PLT Scheme's language framework.\footnote{While some of
% the features are PLT specific, many are supported in some form by
% other Scheme implementations.}
% %
% The purpose of this section is to show how these features are a part
% of a comprehensive language implementation framework, even those that
% appear to be of marginal interest when considered in isolation.

\begin{schemeregion}
\section{Macros}
\label{sect:macros}

PLT Scheme's macro system is based on the
hygienic~\cite{cr:mtw,kffd:hygiene} \scheme{syntax-case}
system~\cite{dhb:sc}, named after the syntactic form it provides for
destructuring the syntax of macro occurrences. A distinguishing aspect
of this system is its use of a syntax object system, a rich datatype
for representing program fragments.

The \scheme|syntax-case| system includes procedural macros, which have
two major advantages over the more widely known pattern-rewriting
macros:

\begin{itemize}
\item
Procedural macros can perform computation at compile time.
\item
Procedural macros allow the programmer to detect and report syntax
errors. Macro writers can enforce constraints on legal syntax (e.g., that
a given list of identifiers must not contain duplicates); detect when
those constraints are violated; and report errors in an appropriate,
context-specific fashion. 
\end{itemize}

The macro definition in Figure~\ref{f:getset} demonstrates the major
capabilities of \scheme{syntax-case} macros. Its purpose is to create
procedures that access and update a shared, hidden variable.
%
For example, a programmer can write
\scheme{(define-getter+setter balance)} to create definitions for
\scheme{get-balance} and \scheme{set-balance!}.

\begin{figure}
\begin{schemedisplay}
(define-syntax (define-getter+setter stx)
  ;; symbol-append : symbol ... $\rightarrow$ symbol
  (define (symbol-append . syms)
    (string->symbol (apply string-append (map symbol->string syms))))
  (syntax-case stx ()
    [(define-getter+setter name init-value)
     ;; constraint checking:
     (unless (identifier? #'name)
       (raise-syntax-error 'define-get+set "expected identifier" #'name))
     ;; transformation:
     (with-syntax
         ([getter (datum->syntax
		   #'name
		   (symbol-append 'get- (syntax->datum #'name)))]
          [setter (datum->syntax
		   #'name
		   (symbol-append 'set- (syntax->datum #'name) '!))])
       #'(define-values (getter setter)
           (let ([name init-value])
             (values (lambda () name)
                     (lambda (new-value) (set! name new-value))))))]))
\end{schemedisplay}
\caption{A \textbf{syntax-case} macro}
\label{f:getset}
\end{figure}

The macro defines a procedural abstraction (\scheme{symbol-append}) to
help construct names. Within the \scheme{syntax-case} clause, the
macro checks that the given name is an identifier (a syntax object
containing a symbol); otherwise, it raises an error. Then it uses the
macro system's \scheme{datum->syntax} procedure together with
its own \scheme{symbol-append} abstraction to construct the names of
the getter and setter procedures. This macro breaks hygiene, because
the hygiene principle states that introduced names only capture
references to the same name that are introduced by the same macro
transformation.
\end{schemeregion}


% The standard \scheme{syntax-case} system defines a few procedures that
%constitute an API to the macro expander, including
%\scheme{datum->syntax} and \scheme{syntax->datum}.

%
%% These, along with the others used in this paper, are given with their
%% contracts in figure \ref{f:api}.

%% \begin{figure}
%% \begin{schemedisplay}
%% ;; syntax-e : syntax $\rightarrow$ datum
%% ;; strips off one layer of syntax object wrapping
%% (syntax-e #'(a b c)) ;$\Rightarrow$ \scheme{(list #'a #'b #'c)}

%% ;; datum$\rightarrow$syntax : syntax datum $\rightarrow$ syntax
%% ;; constructs a new syntax object with the given context
%% (datum->syntax #'here '(+ 1 2)) ;$\Rightarrow$ \scheme{#'(+ 1 2)}

%% \end{schemedisplay}
%% \caption{Macro API}
%% \label{f:api}
%% \end{figure}


\begin{schemeregion}
\section{Modules, or You Want it When, Again?}
\label{sect:modules}

The PLT Scheme module system~\cite{f:modules} allows programmers to
group definitions, use imports and exports to control the
scope of names, and specify the dependencies between modules. The
presence of macros complicates the notion of dependence between
modules.

In the presence of procedural macros, a compiler must execute parts of
a program in order to deal with the remainder of the program. This
blurs the line between compilation and execution.
%
In particular, an interpreter may draw the line in a different place
than the compiler, requiring programmers to debug their compiled
program after they have already debugged their interpreted program. To
eliminate this potential for inconsistency, the PLT Scheme module
system require explicit module dependencies and, based on these, 
provides uniform behavior in both interactive and
batch-compilation mode.

% Modules are compilation units, and every module must be compiled
% before it can be used. Modules contain declarations of their direct
% dependencies. When a module is compiled, the module system uses those
% declarations to determine the portions of existing modules that must
% be executed to support the compilation of the current module. 
% %
% If a macro transformer depends on a value definition, the macro's
% module must declare a ``for-syntax'' dependency on the value
% definition's module.
% %
% Scoping rules prevent access from macros to undeclared run-time
% dependencies, and the compiler creates separate instantiations of
% declared dependencies to prevent interference across separate
% compilations.

\subsection{Split environments}

Syntactically, a module declaration contains a
module reference specifying the language that the module is written
in, the module's name, and a sequence of definitions and expressions. 
In our examples, the module's name is left implicit, and provided in
a comment. In the PLT Scheme implementation, the name is taken from
the filename.
\begin{schemedisplay}
hlang initial-language ;; module-name
module-contents $\cdots$
\end{schemedisplay}
Denotationally, a module consists of two code parts (plus a dependency
specification): a compile-time
component and a run-time component. The compile-time part consists of
the syntax definitions. The run-time part consists of ordinary
definitions and expressions. 

The compiler keeps separate environments for the compile-time
expressions and run-time expressions. If a module defines a procedure
as a run-time value, a macro transformer in the same module cannot
\emph{use} that procedure; the binding is unavailable in the
compile-time phase. The macro can, of course, expand into code that
\emph{refers} to the procedure.  Likewise, a binding in the
compile-time phase cannot be used in the run-time phase. This phase
separation permits the compiler to compile a module without also
executing its entire contents.%
\footnote{The same name may have (possibly distinct) meanings in both phases
 simultaneously. For example, modules written in the \scheme{scheme}
 language automatically import all primitive bindings into both
 phases.}

The two environments yield two kinds of module dependencies and thus two
distinct module import forms. The plain \scheme{require} form imports
bindings into the  environment for run-time expressions, and the
\scheme{for-syntax} variant imports bindings into the
environment for compile-time expressions.

\begin{figure}
\begin{schemedisplay}
langs ;; macro-util
(provide check-for-duplicate-identifier)
(define (check-for-duplicate-identifier ids) ELIDED)

langs ;; rec
(require (for-syntax macro-util))
(define-syntax (recur stx)
  (syntax-case stx ()
    [(recur name ([var init] ...) . body)
     (begin
       (check-for-duplicate-identifier #'(var ...))
       #'(letrec ([name (lambda (var ...) . body)])
	   (name init ...)))]))
(define (build-list n f)
  (recur loop ([i 0])
    (if (< i n)
	(cons (f i) (loop (+ i 1)))
	null)))
\end{schemedisplay}
\caption{Four kinds of references}
\label{fig:four-references}
\end{figure}

Macros bridge the gap between the two phases. The implementation of a
macro is a compile-time expression, but the macro definition extends
the environment for run-time expressions. To understand this idea, it
is important to distinguish between the notions of macro versus value
bindings from the notions of environments for compile-time versus run-time
expressions.

The modules in Figure~\ref{fig:four-references} illustrate the four
different possibilities.
%
In the context of the \scheme{rec} module,
\scheme{check-for-duplicate-identifier} is a value binding in the
compile-time environment; thus, it is available for use in the body of
the \scheme{recur} macro definition. Even though
\scheme{check-duplicate-identifier} is a ``compile-time procedure,''
it is not a macro. In fact, it cannot be used in run-time expressions
at all.
%
In contrast, \scheme{recur} is a macro binding in the run-time
environment. It is bound to a compile-time value, but the
binding is available to run-time expressions such as the definition of
\scheme{build-list}.
%
The occurrence of \scheme{syntax-case} refers to a macro binding in
the compile-time environment. Finally, the definition of
\scheme{build-list} creates a value binding in the run-time
environment.

Compilation of a module involves executing its dependencies%
\footnote{If the module depends on modules that are not already
  compiled, they are automatically compiled when the dependency is
  detected.} and expanding uses of macros in the module's body. The
dependencies include the compile-time part of the module's initial
language module, the compile-time part of every module imported with
\scheme{require}, and both compile-time and run-time parts of every
module imported with \scheme{for-syntax} inside \scheme|require|. 

The rules for compilation (and also for invoking a module's compile
time part) are as follows:
\begin{itemize}
\item
For every \scheme{require} import, including the initial language
module, invoke that module's compile-time part in the same phase.
\item
For every \scheme{for-syntax} import, invoke that module's
compile-time and run-time parts in the next higher phase.
\end{itemize}
If a module is imported twice, once with plain \scheme{require} and once with
\scheme{for-syntax}, the two corresponding invocations of the
module are separate. They do not share mutable state. The module
system uses phase numbers to distinguish the different instances.
%
Finally, a module is invoked only once per phase, per
compilation. Multiple modules that depend on a single module in the
same phase share a single invocation of that module and its state.

\subsection{Compilation independence}

True separate compilation is impossible in a module system that
supports the import and export of macros. Instead, the module system
has a principle of compilation independence:
\begin{quote}
Compiling a module depends only on the compiled forms of the
modules that it (transitively) requires.
\end{quote}
This principle has two consequences:
\begin{itemize}
\item The compilation of two modules, neither of which transitively
  requires the other, should produce the same two results no matter
  which is compiled first, or whether they are compiled in parallel.
\item The compilation of a module does not depend on side effects that
  occurred during the compilation of modules that it transitively
  requires. This has important implications for the use of
  side-effects at compile time.
\end{itemize}

The compiler effectively creates a new store for each module
that it compiles. Each compilation gets a new execution of all
supporting module code.
%
Since the result of the compilation process is nothing but a body of
code, the states of mutable variables and objects created during the
compilation process of any module are discarded at the end.

The pair of modules in figure~\ref{fig:side-effect}
 illustrates the interaction between
side-effects and compilation.
\begin{figure}
\begin{schemedisplay}
langs ;; storage
(define storage '())
(define (add! x) (set! storage (cons x storage)))
(provide storage add!)

langs ;; memory
(require (for-syntax storage))
(define-syntax (remember stx)
  (syntax-case stx ()
    [(remember sym)
     (begin (add! (syntax->datum #'sym))
	    (with-syntax ([syms storage])
              #`(begin (display (quote syms))
		       (newline))))]))
(remember a)
(remember b)
\end{schemedisplay}
\caption{Side Effects and Compilation}
\label{fig:side-effect}
\end{figure}
The first module defines two variables. The second module accesses the
variables at compile time, so it imports the first module via
\scheme{for-syntax}. It defines a \scheme{remember} macro that
adds a symbol to the remembered list and generates code to print out
the updated list of remembered symbols. Then it uses the macro
twice. At the end of compiling the \scheme{memory} module, the
\scheme{storage} variable has the value \schemeresult{(b a)}. Executing the
\scheme{memory} module prints out the lists \schemeresult{(a)} and
\schemeresult{(b a)}, as expected.

Consider the following addition to the program:
\begin{schemedisplay}
langs ;; inspect-storage
(require storage)
(require memory)
(display storage) (newline)
\end{schemedisplay}
When this module is executed, the last line it prints out is
\schemeresult{()}, not \schemeresult{(b a)}, because the
\emph{run-time} instance of the \scheme{storage} module is distinct
from the \emph{compile-time} instance. That is, side-effects do not
cross phases.

Now consider this further addition to the program:
\begin{schemedisplay}
langs ;; more
(require memory)
(remember c)
\end{schemedisplay}
When this module is executed, the last line it prints out is
\schemeresult{(c)}, not \schemeresult{(c b a)}.
%
% This result often surprises macro programmers. Many of them expect the
% final line to be \schemeresult{(c b a)}. It seems to them as if the
% effects in \scheme{memory} occur and are subsequently unwound behind
% their backs. Programming with compile-time side effects can result in
% unexpected behavior---or lack of behavior---unless programmers
% recognize the forgetful nature of the compilation process.
%
The reason that the \scheme{(remember c)} in \scheme{more} prints just 
\schemeresult{(c)} is that \scheme{more} was compiled with a fresh
instance of \scheme{storage} (initially the empty list), and because
executing the compile-time part of \scheme{memory} does not change
that value. The variable is updated during \emph{macro expansion};
the side-effects are not present in the compiled form of
\scheme{memory}:
\begin{schemedisplay}
(compiled-module memory
  (require scheme)
  (require (for-syntax storage))
  (define-syntax (remember stx) ELIDED)
  (begin (display '(a)) (newline))
  (begin (display '(b a)) (newline)))
\end{schemedisplay}
\begin{schemeregion}
{
\newcommand\modulebox[4]{%
\hbox{\framebox{\parbox[t][0.92in][t]{1.0in}{\raggedright{\large #1}\\{\small phase #2}\\{\tiny #4}}}}}

%% \newcommand\modulebox[4]{%
%% \hbox{\framebox{\vbox{\hbox to 1.6in{\large #1}\hbox{\small phase #2 of #3}\hbox{\footnotesize #4}}}}}

\newcommand\modline{\hbox{\hskip 0.5in\vrule height0.1in width1pt}}

\newcommand\cmodulebox[4]{\modulebox{#1}{#2}{}{#4}}
\newcommand\rmodulebox[3]{\modulebox{#1}{#2}{}{#3}}

% parts macros
\newcommand\CT{invoke compile-time part}
\newcommand\EandCT{expand macros, invoke compile-time parts}
\newcommand\CTandRT{invoke compile-time and run-time parts}

\begin{figure}
$$
\begin{array}{cccc}
\mbox{compiling \variablefont{storage}} & \mbox{compiling
  \variablefont{memory}} & 
\mbox{compiling \variablefont{more}} & \mbox{executing
  \variablefont{more}} \\ \hline 
\vbox{%
\cmodulebox{storage}{0}{storage}{\EandCT}%
} &
\vbox{%
\cmodulebox{storage}{1}{memory}{\CTandRT}%
\modline%
\cmodulebox{memory}{0}{memory}{\EandCT}%
} & 
\vbox{%
\vspace{2mm}
\cmodulebox{storage}{1}{more}{\CTandRT}%
\modline%
\cmodulebox{memory}{0}{more}{\CT}%
\modline%
\cmodulebox{more}{0}{more}{\EandCT}%
} &
\vbox{%
\rmodulebox{storage}{1}{\CTandRT}%
\modline%
\rmodulebox{memory}{0}{\CTandRT}%
\modline%
\rmodulebox{more}{0}{\CTandRT}%
} \\
\end{array}
$$
\caption{Module invocations for the execution of \variablefont{more}}%
\label{fig:module-invocations}%
\end{figure}
}
\end{schemeregion}

Figure~\ref{fig:module-invocations} shows all of the module invocations
involved in compiling and executing the program \scheme|more|. Each box represents
a module invocation, and the text at the bottom of each box indicates
what parts of the module are executed. Each column represents a shared
store; effects in one column are not visible in another column.

The furthest left column simply represents the compilation of
\scheme|storage|---this module has no \scheme|for-syntax|
dependencies, and so its compilation triggers no computation in other
modules.  The second column is the compilation of \scheme|memory|,
which requires first running the compile-time portions of the
\scheme|storage| module, since \scheme|memory| requires
\scheme|storage| \scheme|for-syntax|,
then expanding any macros in the \scheme|memory| module. The first two
columns are performed since \scheme|storage| and \scheme|memory| are
both dependencies of \scheme|more|.
Third, the \scheme|more| module is compiled.  This requires running
the compile-time portion of \scheme|memory| (which is
\scheme|require|d by \scheme|more|) and therefore the compile- and run-time
portions of \scheme|storage| (which is \scheme|require|d
\scheme|for-syntax| by \scheme|memory|).  Finally, the fourth column
is the final runtime, which invokes both the compile- and run-time
portions of \scheme|more| and \scheme|memory|, as well as
\scheme|storage|.  

\subsection{Persistent effects}
\label{sect:syntax:persistent}

The compilation rules of the module system require the development of
 a design pattern for expressing persistent effects.
%
Since compile-time side effects are transient, only the code in the compiled
module is permanent. Thus, the way to express a persistent effect is
to make it part of the module:
\begin{schemedisplay}
langs ;; memory.v2
(require (for-syntax storage))
(define-syntax (storage-now stx)
  (syntax-case stx ()
    [(storage-here)
     (with-syntax ([syms storage])
       #'(quote syms))]))
(define-syntax (remember stx)
  (syntax-case stx ()
    [(remember sym)
     #'(begin (define-syntax _ (add! (quote sym)))
	      (display (storage-now))
	      (newline))]))
(remember a)
(remember b)
\end{schemedisplay}
The effect of adding new symbols to the \scheme|storage| variable
 is not executed within the macro, but the macro expander
executes the resulting \scheme{define-syntax} form when it continues
expanding the module body, so the effect of the first addition to the
list still occurs before the second \scheme{remember} is
expanded. This version introduces a helper macro,
\scheme{storage-now}, to retrieve the value of \scheme{storage} after
the update.

Since the compile-time part of a compiled module includes all of the
macro definitions, the side-effect is preserved:
\begin{schemedisplay}
(compiled-module memory.v2
  (require scheme)
  (require (for-syntax storage))
  (define-syntax (storage-now stx) ELIDED)
  (define-syntax (remember stx) ELIDED)
  (define-syntax _1 (add! 'a))
  (display '(a)) (newline)
  (define-syntax _2 (add! 'b))
  (display '(b a)) (newline))
\end{schemedisplay}
The calls to \scheme{add!} are executed whenever
\scheme{memory} is required for the compilation of another
module. Thus they are executed when \scheme{more.v2} is compiled
(refer back to Figure~\ref{fig:module-invocations}), so the storage is
already set to \scheme{(b a)} when the use of \scheme{remember} in
\scheme{more} is expanded. Thus, executing the new version of
the program prints \scheme{(c b a)}.

As a matter of readability, the \scheme{begin-for-syntax} form
accomplishes the same effect as the awkward use of
\scheme{define-syntax} with a throw-away name. Using
\scheme{begin-for-syntax} also explicitly signals the programmer's
intent to generate an expression that creates a persistent effect.

%In summary:
%\begin{itemize}
%\item Persistent effects must be part of the compiled module.
%\item \scheme{require} forms indicate dependence, not just the import
%  of names.
%\item There is no way for one module to affect another module that
%  doesn't depend on it.
%\end{itemize}
\end{schemeregion}


\subsection{Local expansion}

Some special forms must partially expand their bodies before
processing them. For example, primitive forms such as
\scheme{lambda} handle internal definitions by partially expanding
each form in the body to detect whether it is a definition or an
expression. The prefix of definitions is collected and transformed into a
\scheme{letrec} expression with the remainder of the original
forms in the body.

Macros can perform the same kind of partial expansion via the
\scheme{local-expand} procedure, which applies not just to expressions
but to entire modules as well.

\subsection{Compilation-unit hooks}
\label{sect:syntax:hooks}

There are two basic compilation scenarios in PLT Scheme. In
interactive mode, the compiler receives expressions from the
read-eval-print loop. In module mode, the compiler processes an entire
module at once. For each mode, the compiler provides a hook so the
macro system can be used to control compilation of that body of code.

\paragraph{Top-level transformers}

The top-level read-eval-print loop automatically wraps each interaction with the
\scheme{HPtop-interaction} macro. By defining a new version of the
\scheme{HPtop-interaction} macro, a programmer can customize the
behavior of each interaction.

\paragraph{Module transformers}

The macro expander processes a module from top to bottom,
partially expanding to uncover definitions, \scheme{require} and
\scheme{require-for-syntax} forms, and \scheme{provide} forms. It
executes syntax definitions and module import forms as it encounters
them. Then it performs another pass, expanding the remaining run-time
expressions.
%
The module system provides a hook, called \scheme{HPmodule-begin},
that allows language implementations to override the normal expansion
of modules.

The module transformer hook is typically used to constrain the
contents of the module or to automatically import modules into the
compile-time environment. For example, the \scheme{scheme} module
transformer inserts calls to print the values of all top-level
expressions in the module.

The module hook technique has been used before in language
experimentation. Specifically, \citet{pcmkf:continuations} prototyped a language for
programming web servlets using continuations. This prototype was the
first evidence that the module transformer is useful for
general-purpose language experimentation.
%
%FrTime also uses it to optimize functional reactive programs by
%combining nodes in the dataflow evaluation
%graph~\cite{burchett-frtime-opt}.

\begin{schemeregion}
\section{Macro protocols}
\label{sect:protocols}

Some language extensions involve not just a single macro definition, but
a collection of collaborating macros, or one macro whose multiple uses
collaborate. Those collaborating macros need ways to share information
at expansion time.

For example, any datatype created with \scheme{define-struct} can be
recognized and destructured using \scheme{match}, as follows:
\begin{schemedisplay}
(define-struct posn (x y))

(define (dist-to-origin p)
  (match p
    [(struct posn (a b))
     (sqrt (+ (sqr a) (sqr b)))]))
\end{schemedisplay}
The \scheme{define-struct} macro gives \scheme{match} access to the
names of \scheme{posn}'s predicate and accessor functions, and
\scheme{match} uses those names in the expansion of the pattern to
test the value, extract its contents, and bind the results to the
pattern variables \scheme{a} and \scheme{b}.

PLT Scheme provides three mechanisms for compile-time communication between macros:
static bindings, side-effects, and syntax properties. Each mechanism
fits a particular form of communication.

\subsection{Static binding}

PLT Scheme generalizes \scheme{define-syntax} to bind names to
arbitrary compile-time data. The definition of the \scheme{posn}
structure above produces something similar to the following:
\begin{schemedisplay}
(begin
  (define-values (make-posn posn? posn-x posn-y) ELIDED)
  (define-syntax posn
    (list #'make-posn
          #'posn?
          (list #'posn-x #'posn-y))))
\end{schemedisplay}
Despite the use of \scheme{define-syntax}, the definition of
\scheme{posn} is not a macro, as its value is not a transformer
procedure. The static information it carries is accessible from other
macros (such as \scheme{match}) via the \scheme{syntax-local-value}
procedure.

With static binding, the availability of information is tied to the
name it is bound to. Static binding also relies on the ability to
define the name; it cannot attach information to a name that is
already bound.  Still, static binding is the most common mechanism for
defining macro protocols in the PLT Scheme libraries, including
protocols for structs and component
signatures~\cite{culpepper05units}.

\subsection{Side-effects}

Side-effects are commonly used to provide implicit channels of
communication between collaborating run-time components. They are just
as capable of providing such channels at compile time for macros,
provided the programmer recognizes the difference between ephemeral
and persistent effects and uses the appropriate technique.

%%% Discuss ``idempotent'' effects, module-id=? effects

% Due to the interactions between side-effects and the compilation
% process, however, we defer the discussion of side-effects to
% Section~\ref{sect:syntax:modules}.

\subsection{Syntax properties}

\citet{dhb:sc} define a syntax datatype that extends S-expressions
with hygienic binding information and source location tracking.
PLT Scheme adds \emph{syntax properties}, key-value pairs of arbitrary
associated data, as a way of attaching information to particular
terms. By default, syntax properties are simply preserved by macros
and primitive syntactic forms, so protocols defined via syntax
properties generally do not interfere if they choose distinct
keys. Accessing information contained in syntax properties requires
only access to the term that carries the property and the key to the
property. Syntax properties are available even to observers that
cannot access the expansion environment (necessary to access static
bindings and compile-time variables).

For these reasons, syntax properties are well-suited to conveying
information from macros to code analyzers that examine programs after
they have been expanded to core Scheme.
%
For example, DrScheme's Check Syntax tool examines expanded programs to graphically
display the program's binding structure.  This should work even when the reference is no longer apparent in the residual program, as with the  expansion of
\scheme{match}, which uses the information bound to structure name,
although the structure name does not occur in the expansion. The
\scheme{match} macro leaves a \scheme{'disappeared-use} syntax
property on its expansion telling the Check Syntax tool to color the
occurrence of \scheme{posn} as a reference and connect it to the
corresponding definition.

Macros can introduce and examine syntax properties in their arguments
using the \scheme{syntax-property} procedure.

\end{schemeregion}



\end{schemeregion}
